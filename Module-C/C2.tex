\documentclass{beamer}

\usepackage[utf8]{inputenc}
%\usepackage{beamerthemesplit}
\usepackage{url}
\usepackage{tikz}
\usepackage{alltt}
\usepackage{listings}
\usepackage{marvosym}
\usepackage{color}
\usepackage[multidot]{grffile}
\usepackage{multirow}
\usepackage{array}
\usepackage{setspace}
\usepackage{hyperref}
\usepackage{verbatim}
\usepackage{fancyvrb}
%\hypersetup{colorlinks=true, linkcolor=blue,  anchorcolor=blue,  
%citecolor=blue, filecolor=blue, menucolor=blue, pagecolor=blue,  
%urlcolor=blue} 
\lstset{keywordstyle=\bfseries\color{brown},
        stringstyle=\ttfamily,
        commentstyle=\color{blue}\textit,
        showstringspaces=false}

\useoutertheme{}
\usetheme{Madrid}
\graphicspath{{pics/}{global/}
{pics/I/}{pics/A1/}{pics/A2/}{pics/A3/}{pics/A4/}{pics/A5/}{pics/A6/}{pics/A7/}
}

\logo{\includegraphics[height=1cm]{ProcessHorizontal}} 

\institute{Center for Computation and Technology\\Louisiana State University, Baton Rouge, LA}

\setbeamertemplate{navigation symbols}{} 

\title{CSC 7700: Scientific Computing}

% We want to use the infolines outer theme because it does not use a lot of
% space, but it also tries to print an institution and the slide
% numbers (which we might not want to show). Therefore, we here redefine the
% footline ourselfes - mostly a copy & paste from
% /usr/share/texmf/tex/latex/beamer/themes/outer/beamerouterthemeinfolines.sty
\defbeamertemplate*{footline}{infolines theme without institution and slide numbers}
{
  \leavevmode%
  \hbox{%
  \begin{beamercolorbox}[wd=.25\paperwidth,ht=2.25ex,dp=1ex,center]{author in head/foot}%
    \usebeamerfont{author in head/foot}\insertshortauthor
  \end{beamercolorbox}%
  \begin{beamercolorbox}[wd=.5\paperwidth,ht=2.25ex,dp=1ex,center]{title in head/foot}%
    \usebeamerfont{title in head/foot}\insertshorttitle
  \end{beamercolorbox}%
  \begin{beamercolorbox}[wd=.25\paperwidth,ht=2.25ex,dp=1ex,center]{date in head/foot}%
    \usebeamerfont{date in head/foot}\insertshortdate{}
  \end{beamercolorbox}}%
  \vskip0pt%
}

% Some useful commands
\newcommand{\abspic}[4]
 {\vspace{ #2\paperheight}\hspace{ #3\paperwidth}\includegraphics[height=#4\paperheight]{#1}\\
  \vspace{-#2\paperheight}\vspace{-#4\paperheight}\vspace{-0.0038\paperheight}}

\newcommand{\picw}[4]{{
 \usebackgroundtemplate{
 \color{black}\vrule width\paperwidth height\paperheight\hspace{-\paperwidth}\hspace{-0.01\paperwidth}
 \hspace{#4\paperwidth}\includegraphics[width=#3\paperwidth, height=\paperheight]{#1}}\logo{}
 \frame[plain]{\frametitle{#2}}
}}
\newcommand{\pic}[2]{\picw{#1}{#2}{}{0}}

\newcommand{\question}[1]{\frame{\frametitle{#1}
 \begin{centering}\Huge #1\\\end{centering}
}}


\usepackage{wrapfig}

%\usecolortheme[RGB={200,200,200}]{structure}

\subtitle[Module C]{{\large Module C: Advanced Programming Tools}\\*[0.3em]Lectures 3/4: Eclipse IDE}
\author[Dr. Steven R. Brandt]{Dr. Steven R. Brandt}
\date{September 28, 2012}
%\usecolortheme[RGB={160,45,140}]{structure}
\usecolortheme[RGB={110,65,240}]{structure}

\begin{document}

\frame{\titlepage}

\section*{Outline}
\frame{\tableofcontents}

\section{Goals}
\frame{\frametitle{}\begin{centering}\LARGE\insertsectionhead\\\end{centering}}

\frame[containsverbatim]{ \frametitle{Goals}
  \begin{itemize}
    \item  The module \emph{Advanced Programming Tools} will teach:
    \begin{itemize}
      \item Eclipse
      \item Installing Eclipse and the JDK
      \item Views, Editors, Codes
      \item Tools for Static Analysis
      \item Tools for Debugging
      \item Linux Tools for Eclipse
      \item HPC Toolkit Plugin
      \item Mojave
    \end{itemize}
    \item We will use Cactus as an example of an Application Framework.
  \end{itemize}
}

\section{Eclipse}
\frame{\frametitle{}\begin{centering}\LARGE\insertsectionhead\\\end{centering}}

\frame{ \frametitle{Eclipse}
    \begin{center}
    \includegraphics[width=5cm]{pics/eclipselogo}\mbox{\hspace{0.5em}}
    \end{center}
    \begin{itemize}
        \item {IDE: Integrated Development Environment}
        \item {Advanced editing capabilities}
        \item {Written in Java for multi-platform support}
        \item {open-source}
        \item {extensible}
        \item {Supports C, C++, Fortran, etc. through ``plugins''}
    \end{itemize}
}

\subsection{Installing the Oracle JDK}
\frame{\frametitle{}\begin{centering}\LARGE\insertsubsectionhead\\\end{centering}}

\frame{ \frametitle{Installing the Oracle JDK}
    \begin{center}
    \includegraphics[width=3cm]{pics/getjava}\mbox{\hspace{0.5em}}
    \end{center}
\begin{itemize}
\item {Goto http://java.sun.com}
\item {Click on ``Java SE'' under ``Top Downloads''}
\item {Click on the ``Java Download'' button}
\item {Scroll down and get the Java SE 6 JDK (Java Development Kit)}
\end{itemize}
}

\frame[containsverbatim]{ \frametitle{Installing the Oracle JDK}
After you run the installer, add the following to your .bashrc
\begin{itemize}
\item \verb=export JAVA_HOME==\verb=/usr/java/jdk1.6.0_27=
\item \verb=export PATH==\verb=$JAVA_HOME/bin:$PATH=
\item Now source your .bashrc
\end{itemize}
}

\subsection{Installing Eclipse}
\frame{\frametitle{}\begin{centering}\LARGE\insertsubsectionhead\\\end{centering}}

\frame[containsverbatim]{ \frametitle{Installing Eclipse}
\begin{itemize}
\item Goto \verb=http://www.eclipse.org/downloads/=
\item Get Eclipse for Parallel Application Developers
\item Command: {\small\verb=tar xvf ~/Download/eclipse-SDK-4.2-linux-gtk-x86_64.tar.gz=}
\item Command: \verb=cd eclipse/=
\item Command: \begin{verbatim}
./eclipse -Xms1024m -Xmx2048m \
    -XX:PermSize=256m -XX:MaxPermSize=512m &
\end{verbatim}
\item You'll see a prompt for selecting a workspace. Check the box which says ``Use this as the default and do not ask again''
\end{itemize}
}

%\frame[containsverbatim]{ \frametitle{Installing Eclipse/PTP}
%\begin{itemize}
%\item Help \verb=>= Install new software...
%\item Where it says "Work with" select ``--All Available Sites--''
%\item Then click the blue link which says ``Available Software Sites''
%\item In the popup that appears, type ``ptp'' in the filter for available software sites
%\item Check the one item that is selected
%\item Hit OK
%\item Hit ``Select All''
%\end{itemize}
%}

%\frame[containsverbatim]{ \frametitle{Installing Eclipse/PTP}
%    \begin{center}
%    \includegraphics[width=9cm]{pics/install1}\mbox{\hspace{0.5em}}
%    \end{center}
%}

%\frame[containsverbatim]{ \frametitle{Installing Eclipse/PTP}
%\begin{itemize}
%\item Where it says "Work with" select ``http://download.eclipse.org/tools/ptp/update''
%\item Hit ``Select All''
%\item Hit ``Next \verb=>=''
%\item Hit ``Next \verb=>=''
%\item Accept the license agreement
%\item Hit ``Finish''
%\item Hit ``Restart Now''
%\item In the future, to update Eclipse go to {\tt Help $>$ Update Software}
%\end{itemize}
%}

\section{Hello World in C}
\frame{\frametitle{}\begin{centering}\LARGE\insertsectionhead\\\end{centering}}

\frame[containsverbatim]{ \frametitle{Hello World in C}
\begin{itemize}
\item File \verb=>= New \verb=>= C Project (or select ``Project
 Explorer'' and type Shift-Alt-N))
\item Select ``Hello World ANSI C Project''
\item Fill in project name
\item Click ``Next''
\end{itemize}
\begin{center}
    \includegraphics[width=4cm]{pics/hello}\mbox{\hspace{0.5em}}
\end{center}
}

\frame[containsverbatim]{ \frametitle{Hello World in C}
``Project \verb=>= Build Project'' builds a project
\begin{center}
    \includegraphics[width=7cm]{pics/buildhello}\mbox{\hspace{0.5em}}
\end{center}
}

\frame[containsverbatim]{ \frametitle{Hello World in C}
``Run \verb=>= Run'' runs the program
\begin{center}
    \includegraphics[width=7cm]{pics/runhello}\mbox{\hspace{0.5em}}
\end{center}
}

\subsection{Advanced Editing}
\frame{\frametitle{}\begin{centering}\LARGE\insertsubsectionhead\\\end{centering}}

\frame[containsverbatim]{ \frametitle{Control Sequences}
\begin{itemize}
\item Find source definition (Ctrl-G to show, F3 to go)
\item Alt-$\leftarrow$ navigates to the previous source window (like
 a web browser back button)
\item Shift-Ctrl-G find symbol in workspace
\item Ctrl-H to search
\item Alt-Shift-$\leftarrow$ or Alt-Shift-$\rightarrow$ highlights a block
\end{itemize}
}
\frame[containsverbatim]{ \frametitle{Control Sequences}
\begin{itemize}
\item Shift-Ctrl-P bounces from the start to the end of a block (or vice versa)
\item Correct indentation: highlight region, then Ctrl-I
\item Toggle comment: Ctrl-/
\item Correct format: highlight region, then Ctrl-Shift-F
\item But what is the correct style?
\end{itemize}
}

\frame[containsverbatim]{ \frametitle{Project Style}
\begin{itemize}
\item Select a project
\item Select: Project $>$ Properties
\item Open C/C++ General
\item Select ``Formatter''
\end{itemize}
\begin{center}
    \includegraphics[width=6cm]{pics/estyle}\mbox{\hspace{0.5em}}
\end{center}
}

\frame[containsverbatim]{ \frametitle{Refactorings}
\begin{itemize}
\item Alt-Shift-R rename variable or function
\item Alt-Shift-L extract local variable
\item Alt-Shift-M extract method
\item Surround with...
\end{itemize}
}

%\frame[containsverbatim]{ \frametitle{Code Templates}
%\begin{center}
%    \includegraphics[width=6cm]{pics/codet}\mbox{\hspace{0.5em}}
%\end{center}
%}

\frame[containsverbatim]{ \frametitle{Perspective / Editor / View}
\begin{itemize}
\item A perspective is a collection of views. You can change
them by going to \verb=Window > Open Perspective...=
\item On the right you see the outline view.
\item Click the small x and it will go away.
\item To bring it back, use \verb=Window > Open View...= (or Shift-Alt-Q)
\end{itemize}
}

\frame[containsverbatim]{ \frametitle{History}
\begin{itemize}
\item The ``History'' view tracks session edits (Shift-Alt-Q Z)
\item It can recover old versions from within the session
\item By default, you won't see edits from a previous session,
    but if you click the ``Link with Editor and Selection'' you
    can see them. \includegraphics[width=0.5cm]{pics/linkwith}
\item Old versions stored in \verb=.metadata/.plugins/org.eclipse.core.resources/.history=
\end{itemize}
\begin{center}
    \includegraphics[width=6cm]{pics/history}\mbox{\hspace{0.5em}}
\end{center}
}

%\frame[containsverbatim]{ \frametitle{Preferences}
%\begin{itemize}
%\item Eclipse rebuilds all open projects before running.
%    Very annoying.
%\item To turn it off, go to \verb=Window > Preferences=
%\item Open Run/Debug
%\item Select Launching
%\item Uncheck ``Build (if required) before launching''
%\end{itemize}
%\begin{center}
%    \includegraphics[width=6cm]{pics/bifl}\mbox{\hspace{0.5em}}
%\end{center}
%}

\frame[containsverbatim]{ \frametitle{Automatic Updates}
\begin{center}
    \includegraphics[width=6cm]{pics/autou}\mbox{\hspace{0.5em}}
\end{center}
}


\subsection{Linux Tools}
\frame{\frametitle{}\begin{centering}\LARGE\insertsubsectionhead\\\end{centering}}

% Linux Tools - http://downloads.eclipse.org/technology/linuxtools/update
% gcov
% gprof
% valgrind
%\frame[containsverbatim]{ \frametitle{LinuxTools}
%\begin{itemize}
%\item Help $>$ Install New Software...
%\item Press ``Add'' set ``Name:'' to ``Linux Tools'' and Location to ``http://downloads.eclipse.org/technology/linuxtools/update''
%\item Hit ``Next'' etc.
%\end{itemize}
%}
\frame[containsverbatim]{ \frametitle{LinuxTools: Valgrind}
\begin{figure}
\includegraphics[width=7cm]{pics/valcfg}
\end{figure}
}
\frame[containsverbatim]{ \frametitle{LinuxTools: Valgrind}
\begin{figure}
\includegraphics[width=7cm]{pics/valuse}
\end{figure}
}
%\frame[containsverbatim]{ \frametitle{LinuxTools: OProfile Setup}
%Step 1:
%\begin{center}
%\includegraphics[width=6cm]{pics/opcfg}\\
%\end{center}
%Step 2:
%\begin{center}
%\includegraphics[width=6cm]{pics/opcfg2}
%\end{center}
%}

%\frame[containsverbatim]{ \frametitle{LinuxTools: OProfile}
%The result of running oprofile is that the oprofile view
%gives you information about your run. You can click inside
%the view and see the relevant line in the editor.
%\begin{figure}
%\includegraphics[width=6cm]{pics/opuse}
%\end{figure}
%}

\frame[containsverbatim]{ \frametitle{LinuxTools: gprof}
\begin{itemize}
\item Revise makefile to use CC, CFLAGS, and LDFLAGS
\item Select Project and right click
\item Project Properties $>$ C/C++ Build $>$ Environment
\item Click on "Add"
\item Set ``name'' to ``CFLAGS'' and ``value'' to ``-g -pg''
\item Compile and run
\item Linux Tools can process gmon.out
\end{itemize}
\begin{figure}
    \includegraphics[width=5cm]{pics/gprof}
\end{figure}
}

\subsection{Hello World in C+MPI}
\frame{\frametitle{}\begin{centering}\LARGE\insertsubsectionhead\\\end{centering}}

\frame[containsverbatim]{ \frametitle{Hello World in C+MPI}
Upgrading the program to use MPI
\begin{itemize}
\item Add \verb=#include <mpi.h>=
\item Click inside the program type ``mpi'' then hit Ctrl-space.
  You'll see a code completion options. Choose ``MPI Init and Finalize''
  you now have errors in your code. Use the editor to add argc and argv to main.
\end{itemize}
\begin{center}
    \includegraphics[width=6cm]{pics/helloupgrade}\mbox{\hspace{0.5em}}
\end{center}
}

\frame[containsverbatim]{ \frametitle{Hello World in C+MPI}
Building will now produce errors.
\begin{center}
    \includegraphics[width=7cm]{pics/hellofail}\mbox{\hspace{0.5em}}
\end{center}
}

\frame[containsverbatim]{ \frametitle{Hello World in C+MPI}
To resolve these errors...
\begin{itemize}
\item Right click on the ``Hello'' project in the project explorer view.
\item Select ``Properties''
\item Open ``C/C++ General'' and click on ``Paths and Symbols''
\item Select the C language and click ``Add'' to add the include.
\end{itemize}
\begin{center}
    \includegraphics[width=10cm]{pics/hellofix}\mbox{\hspace{0.5em}}
\end{center}
}

\frame[containsverbatim]{ \frametitle{Hello World in C+MPI}
How did I know the mpi include path?
\begin{itemize}
\item Run \verb=sh -x mpicc test.c=
\item Output will contain something like this:
    + gcc -m64 -O2 -fPIC -Wl,-z,noexecstack test.c -I/usr/include/mpich2-x86\_64 -L/usr/lib64/mpich2/lib -L/usr/lib64/mpich2/lib -lmpich -lopa -lpthread -lrt
\item Now you can see the include (/usr/include/mpich2-x86\_64), the lib dir (/usr/lib64/mpich2/lib), libs, etc.
\end{itemize}
}

\frame[containsverbatim]{ \frametitle{Hello World in C+MPI}
\begin{itemize}
\item While you're still in ``C/C++ General'', click on the ``Library Paths'' tab and fix that. (In my case, add /usr/lib64/mpich2/lib)
\item Next, go to the ``Libraries'' tab and fix that
\end{itemize}
\begin{center}
    \includegraphics[width=10cm]{pics/hellofix2}\mbox{\hspace{0.5em}}
\end{center}
}

\frame[containsverbatim]{ \frametitle{Hello World in C+MPI}
%\begin{wrapfigure}[10]{r}{1cm}
%    \includegraphics[width=5cm]{pics/hellorm}\mbox{\hspace{0.5em}}
%\end{wrapfigure}
\begin{itemize}
\item ``Project \verb=>= Build'' should now work again.
\item Now we need to set up a parallel run.
Click ``Window \verb=>= Open Perspective \verb=>= Other... \verb=>= Parallel Runtime''
\item Right click inside the ``Resource Managers'' tab.
\item Click on ``Add Resource Manager...''
\item Choose either MPICH2 or OpenMP, whichever you have on your machine.
\item Click Next \verb=>= Finish.
\item In a separate window, type mpdboot to start mpich2
\item Right click on the Resource manager and select start.
\end{itemize}
}

\frame{ \frametitle{Hello World in C+MPI}
\begin{figure}
    \includegraphics[width=5cm]{pics/hellorm}\mbox{\hspace{0.5em}}
    \includegraphics[width=5cm]{pics/jobmgr}\mbox{\hspace{0.5em}}
\end{figure}
}

\frame[containsverbatim]{ \frametitle{Hello World in C+MPI}
\begin{figure}
\begin{itemize}
\item Click \verb=Run > Run Configurations...=
\item Right Click on \verb=Parallel Applications= and select \verb=New=
\item Select the new application, configure the ``Application program.''
\item Click ``Apply'' and ``Run''
\end{itemize}
    \includegraphics[width=6cm]{pics/newpar2}\mbox{\hspace{0.5em}}
\end{figure}
}

\frame[containsverbatim]{ \frametitle{Hello World in C+MPI}
\begin{figure}
\begin{itemize}
\item Click \verb=Run > Run Configurations...=
\item Select your parallel application
\item Select the ``Resources'' tab
\item Adjust the ``Number of processes'' to 2
\item Click ``Apply'' and ``Run''
\end{itemize}
    \includegraphics[width=5cm]{pics/newpar3}\mbox{\hspace{0.5em}}
    \includegraphics[width=5cm]{pics/runpar3}\mbox{\hspace{0.5em}}
\end{figure}
}

\subsection{Adding Message Passing}
\frame{\frametitle{}\begin{centering}\LARGE\insertsubsectionhead\\\end{centering}}

\frame[containsverbatim]{ \frametitle{Hello World in C+MPI}
\begin{itemize}
\item Naviagate back to the C/C++ perspective (\verb=Window > Open Perspective > C/C++=)
\item Click in the code editor between \verb=MPI_Init= and \verb=MPI_Finalize=.
\item Type ``mpi'' and hit Ctrl-space. Take the ``mpisr'' code completion. A complete skeleton for doing an MPI send and receive will appear. Fill in the missing variable declarations.
\begin{verbatim}
    int rank,p,source,dest,tag=66;
    MPI_Status status;
    char message[100];
\end{verbatim}
\item Alt-P followed by B will build the current project.
\item Click on the down arrow and select your run config.
    \includegraphics[width=1cm]{pics/rundown}\mbox{\hspace{0.5em}}
\end{itemize}
}

\subsection{Parallel Analysis Tools}
\frame{\frametitle{}\begin{centering}\LARGE\insertsubsectionhead\\\end{centering}}

\frame[containsverbatim]{ \frametitle{Using Parallel Analysis}
\begin{itemize}
\item The most basic form of parallel analysis is available through the
    parallel analysis button \includegraphics[width=0.5cm]{pics/panbut}
\item Select a source file in the package explorer
\item Click the down arrow, then choose ``Show MPI Artifacts''
\end{itemize}
\begin{figure}
    \includegraphics[width=7cm]{pics/panan}
\end{figure}
}

\frame[containsverbatim]{ \frametitle{Using Parallel Analysis}
\begin{itemize}
\item Select a source file in the package explorer
\item Click the down arrow, then choose ``MPI Barrier Analysis''
\end{itemize}
\begin{figure}
    \includegraphics[width=7cm]{pics/panpban}
\end{figure}
}

\frame[containsverbatim]{ \frametitle{Using Parallel Analysis}
\begin{itemize}
\item Now introduce a barrier error
\item Select a source file in the package explorer
\item Click the down arrow, then choose ``MPI Barrier Analysis''
\end{itemize}
\begin{figure}
    \includegraphics[width=7cm]{pics/panpbanx}
\end{figure}
}

%\frame[containsverbatim]{ \frametitle{Using GEM}
%\begin{itemize}
%\item To use the GEM tools, you must first install isp
%\item Isp is a tool from University of Utah to analyze C/C++-language
%      programs and discover MPI errors.
%\item It contains a compiler (ispcc), a tool to run a program (isp), and a viewer. This tool may be used outside of Eclipse.
%\item Installation of isp follows the familiar pattern of configure / make / make install. Make sure the ispcc command is in your path when you start Eclipse.
%\item Download page: \verb=http://www.cs.utah.edu/formal_verification/ISP-release/=
%\end{itemize}
%}

%\frame[containsverbatim]{ \frametitle{Using GEM}
%\begin{itemize}
%\item To run GEM in Eclipse, click the trident.
%    \includegraphics[width=1cm]{pics/trident}
%\item You will be prompted for command line arguments. Click ``OK''.
%    \includegraphics[width=3cm]{pics/gemcmdline}
%\item GEM will now analyze your code. Click the ``GEM Analyzer'' tab, then ``Browse MPI Calls.''
%\end{itemize}
%\begin{figure}
%    \includegraphics[width=5cm]{pics/gemout}
%\end{figure}
%}

%\frame[containsverbatim]{ \frametitle{Using GEM}
%\begin{itemize}
%\item Now let's introduce an error. A call to \verb+MPI_Barrier()+ just
%before the call to \verb=MPI_Send()=
%\item Save the file. Build the project. Run GEM. Deadlock detected.
%\end{itemize}
%\begin{figure}
%    \includegraphics[width=5.5cm]{pics/gemdead}
%    \mbox{\vspace{0.5em}}
%    \includegraphics[width=5.5cm]{pics/gemdead2}
%\end{figure}
%}

% Linux Tools
% Building with Mojave
% Submitting to a remote machine with the SimFactory resource
% Need to talk about generic eclipse stuff at some point
%   Ctrl-R, search, scroll size, Ctrl-space, refactoring
% Linux tools? Are they part of PTP?
% Mojave
% Mojave menu

\section{Mojave}
\frame{\frametitle{}\begin{centering}\LARGE\insertsectionhead\\\end{centering}}

\frame[containsverbatim]{ \frametitle{Mojave}
\begin{itemize}
\item Mojave: A place for Cacti to live
\item An interface to the Cactus Source Code
\item An interface to SimFactory
\end{itemize}
}

\subsection{Installing Mojave}
\frame{\frametitle{}\begin{centering}\LARGE\insertsubsectionhead\\\end{centering}}

\frame[containsverbatim]{ \frametitle{Installing Mojave}
\begin{itemize}
\item Help $>$ Install New Software $>$ Add...
\item A popup will appear. Fill in the Name and Location
\item Hit ``OK'' then ``Next'' and accept, etc. on the following screens.
\end{itemize}
\begin{figure}
    \includegraphics[width=7cm]{pics/minstall}
    \mbox{\vspace{0.5em}}
\end{figure}
}

\subsection{The Wave Equation on your Local Machine}
\frame{\frametitle{}\begin{centering}\LARGE\insertsubsectionhead\\\end{centering}}

\frame[containsverbatim]{ \frametitle{Mojave}
\begin{itemize}
\item To create a Mojave Project, start by selecting
\verb=File > New Project > Other... > Fortan > Fortran Project=
\item Select an empty makefile project
\end{itemize}
\begin{figure}
    \includegraphics[width=5.0cm]{pics/fortranwiz}
\end{figure}
}
\frame[containsverbatim]{ \frametitle{Mojave}
\begin{itemize}
\item When you get to the last screen of the wizard, you
    will be prompted to create one of three types of projects.
\item Select the ``Mojave Download Project'' and create a WaveDemo
\end{itemize}
\begin{figure}
    \includegraphics[width=5.0cm]{pics/mojavewiz}
\end{figure}
}

\frame[containsverbatim]{ \frametitle{Mojave}
\begin{itemize}
\item Now that you've downloaded a project, please configure using
the mojave variable editor
\item \verb=Mojave > Edit Variables...=
\item Select the basic information needed for a cactus build/run
\end{itemize}
\begin{figure}
    \includegraphics[width=7cm]{pics/mojavevars}\mbox{\vspace{0.5em}}
\end{figure}
}

\frame{ \frametitle{Mojave}
\begin{itemize}
    \item \textcolor{blue}{Mojave $>$ Build}
    \item Mojave $>$ CreateSim
    \item Mojave $>$ RunSim
\end{itemize}
\begin{figure}
    \includegraphics[width=7cm]{pics/mbuild}\mbox{\vspace{0.5em}}
\end{figure}
}

\frame{ \frametitle{Mojave}
\begin{itemize}
    \item Mojave $>$ Build
    \item \textcolor{blue}{Mojave $>$ CreateSim}
    \item Mojave $>$ RunSim
\end{itemize}
\begin{figure}
    \includegraphics[width=7cm]{pics/mcreatesim}\mbox{\vspace{0.5em}}
\end{figure}
}

\frame{ \frametitle{Mojave}
\begin{itemize}
    \item Mojave $>$ Build
    \item Mojave $>$ CreateSim
    \item \textcolor{blue}{Mojave $>$ RunSim}
\end{itemize}
\begin{figure}
    \includegraphics[width=7cm]{pics/mrunsim}\mbox{\vspace{0.5em}}
\end{figure}
}

\frame{ \frametitle{Mojave}
\begin{itemize}
\item Now to use ranger.
\item Change variable definitions in Mojave.
\item Make sure defs.local.ini is correct (you can edit it in Mojave as well, just use Ctrl-R to call up the file)
\item My ranger configuration looks like this:\\
{\tt
[ranger]\\
user = tg457049\\
sourcebasedir = /work/00044/@USER@\\
basedir = /scratch/00044/@USER@/simulations
}
\end{itemize}
\begin{figure}
    \includegraphics[width=4cm]{pics/rangervars}\mbox{\vspace{0.5em}}
\end{figure}
}

\subsection{The Wave Equation on Ranger}
\frame{\frametitle{}\begin{centering}\LARGE\insertsubsectionhead\\\end{centering}}

% on ranger
\frame{ \frametitle{Mojave}
\begin{itemize}
    \item \textcolor{blue}{Mojave $>$ Build}
    \item Mojave $>$ CreateSim
    \item Mojave $>$ SubmitSim
\end{itemize}
\begin{figure}
    \includegraphics[width=7cm]{pics/rbuild}\mbox{\vspace{0.5em}}
\end{figure}
}

\frame{ \frametitle{Mojave}
\begin{itemize}
    \item Mojave $>$ Build
    \item \textcolor{blue}{Mojave $>$ CreateSim}
    \item Mojave $>$ SubmitSim
\end{itemize}
\begin{figure}
    \includegraphics[width=7cm]{pics/rcreatesim}\mbox{\vspace{0.5em}}
\end{figure}
}

\frame{ \frametitle{Mojave}
\begin{itemize}
    \item Mojave $>$ Build
    \item Mojave $>$ CreateSim
    \item \textcolor{blue}{Mojave $>$ SubmitSim}
\end{itemize}
\begin{figure}
    \includegraphics[width=7cm]{pics/rsubmit}\mbox{\vspace{0.5em}}
\end{figure}
}

\frame{ \frametitle{Mojave}
\begin{itemize}
    \item Mojave $>$ Build
    \item Mojave $>$ CreateSim
    \item Mojave $>$ SubmitSim
    \item \textcolor{blue}{Mojave $>$ ListSims}
\end{itemize}
\begin{figure}
    \includegraphics[width=7cm]{pics/rlist}\mbox{\vspace{0.5em}}
\end{figure}
}

\frame{ \frametitle{Mojave}
\begin{itemize}
    \item Mojave $>$ Build
    \item Mojave $>$ CreateSim
    \item Mojave $>$ SubmitSim
    \item Mojave $>$ ListSims
    \item \textcolor{blue}{Mojave $>$ ShowSimOutput}
\end{itemize}
\begin{figure}
    \includegraphics[width=7cm]{pics/routput}\mbox{\vspace{0.5em}}
\end{figure}
}
%on ranger

\frame[containsverbatim]{ \frametitle{Mojave}
Using the Mojave Menu
\begin{itemize}
\item The Mojave menu comes pre-configured with all the basic
commands you might want to run.
\item ``New Thorn'' allows you to add to the existing set of thorns.
\item ``Update Repo'' will invoke GetComponents to update the existing
    installation.
\item Build provides an alternative to the \verb=Project > Build= menu.
\item CreateSim will create a simfactory run configuration
\item CleanupSim will call simfactory cleanup on a run configuration
\item PurgeSim will erase the data associated with a simfactory run
\end{itemize}
}

\frame[containsverbatim]{ \frametitle{Mojave}
Advanced Configuration of the Mojave Menu
\begin{itemize}
\item Type Ctrl-R, then .mojave.xml.
\item Before you finish typing,
Eclipse will find the file and complete the name for you. Pull
it up in the editor.
\item The file name will show in matching items.
\end{itemize}
\begin{figure}
    \includegraphics[width=7cm]{pics/filecomplete}\mbox{\vspace{0.5em}}
\end{figure}
}

\frame[containsverbatim]{ \frametitle{Mojave}
\begin{itemize}
\item .mojave.xml contains variables and actions.
\item actions are made up of a sequence of commands
\item commands can have a name, or a name value pair
\item if a value is not defined, neither the name nor value will
      be placed into the generated shell command
\item editing .mojave.xml directly can be useful for adding
      or changing variables, or adding or changing menu items.
\end{itemize}
}

\frame[containsverbatim]{ \frametitle{Mojave}
\begin{figure}
    \includegraphics[width=7cm]{pics/mojavexml}\mbox{\vspace{0.5em}}
\end{figure}
}

\frame[containsverbatim]{ \frametitle{Mojave}
\begin{itemize}
\item Commands can be conditional on a variable definition
\item Multiple commands can be part of a single action with
    combined console output.
\end{itemize}
\begin{figure}
    \includegraphics[width=7cm]{pics/cleanbuild}\mbox{\vspace{0.5em}}
\end{figure}
}

\frame[containsverbatim]{ \frametitle{Mojave}
\begin{itemize}
    \item To update symbols after a cactus build, right click on
       the project then select \verb=Index > Rebuild=
    \item Eclipse will then update all its symbol information with
       the contents of the files generated during the build.
    \item By default, Mojave builds cactus files inside the
        \verb=~/.mojaveconfig= directory.
\end{itemize}
\begin{figure}
    \includegraphics[width=7cm]{pics/defines}\mbox{\vspace{0.5em}}
\end{figure}
}

\end{document}
