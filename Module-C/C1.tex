\documentclass{beamer}

\usepackage[utf8]{inputenc}
%\usepackage{beamerthemesplit}
\usepackage{url}
\usepackage{tikz}
\usepackage{alltt}
\usepackage{listings}
\usepackage{marvosym}
\usepackage{color}
\usepackage[multidot]{grffile}
\usepackage{multirow}
\usepackage{array}
\usepackage{setspace}
\usepackage{hyperref}
\usepackage{verbatim}
\usepackage{fancyvrb}
%\hypersetup{colorlinks=true, linkcolor=blue,  anchorcolor=blue,  
%citecolor=blue, filecolor=blue, menucolor=blue, pagecolor=blue,  
%urlcolor=blue} 
\lstset{keywordstyle=\bfseries\color{brown},
        stringstyle=\ttfamily,
        commentstyle=\color{blue}\textit,
        showstringspaces=false}

\useoutertheme{}
\usetheme{Madrid}
\graphicspath{{pics/}{global/}
{pics/I/}{pics/A1/}{pics/A2/}{pics/A3/}{pics/A4/}{pics/A5/}{pics/A6/}{pics/A7/}
}

\logo{\includegraphics[height=1cm]{ProcessHorizontal}} 

\institute{Center for Computation and Technology\\Louisiana State University, Baton Rouge, LA}

\setbeamertemplate{navigation symbols}{} 

\title{CSC 7700: Scientific Computing}

% We want to use the infolines outer theme because it does not use a lot of
% space, but it also tries to print an institution and the slide
% numbers (which we might not want to show). Therefore, we here redefine the
% footline ourselfes - mostly a copy & paste from
% /usr/share/texmf/tex/latex/beamer/themes/outer/beamerouterthemeinfolines.sty
\defbeamertemplate*{footline}{infolines theme without institution and slide numbers}
{
  \leavevmode%
  \hbox{%
  \begin{beamercolorbox}[wd=.25\paperwidth,ht=2.25ex,dp=1ex,center]{author in head/foot}%
    \usebeamerfont{author in head/foot}\insertshortauthor
  \end{beamercolorbox}%
  \begin{beamercolorbox}[wd=.5\paperwidth,ht=2.25ex,dp=1ex,center]{title in head/foot}%
    \usebeamerfont{title in head/foot}\insertshorttitle
  \end{beamercolorbox}%
  \begin{beamercolorbox}[wd=.25\paperwidth,ht=2.25ex,dp=1ex,center]{date in head/foot}%
    \usebeamerfont{date in head/foot}\insertshortdate{}
  \end{beamercolorbox}}%
  \vskip0pt%
}

% Some useful commands
\newcommand{\abspic}[4]
 {\vspace{ #2\paperheight}\hspace{ #3\paperwidth}\includegraphics[height=#4\paperheight]{#1}\\
  \vspace{-#2\paperheight}\vspace{-#4\paperheight}\vspace{-0.0038\paperheight}}

\newcommand{\picw}[4]{{
 \usebackgroundtemplate{
 \color{black}\vrule width\paperwidth height\paperheight\hspace{-\paperwidth}\hspace{-0.01\paperwidth}
 \hspace{#4\paperwidth}\includegraphics[width=#3\paperwidth, height=\paperheight]{#1}}\logo{}
 \frame[plain]{\frametitle{#2}}
}}
\newcommand{\pic}[2]{\picw{#1}{#2}{}{0}}

\newcommand{\question}[1]{\frame{\frametitle{#1}
 \begin{centering}\Huge #1\\\end{centering}
}}



%\usecolortheme[RGB={200,200,200}]{structure}

\subtitle[Module C]{{\large Module C: Advanced Programming Tools}\\*[0.3em]Lectures 1/2: Command Line Tools}
\author[Dr. Steven R. Brandt]{Dr. Steven R. Brandt}
\date{September 20, 2013}
%\usecolortheme[RGB={160,45,140}]{structure}
\usecolortheme[RGB={80,45,200}]{structure}

\begin{document}

\frame{\titlepage}

\section*{Outline}
\frame{\tableofcontents}

\section{Goals}
\frame{\frametitle{}\begin{centering}\LARGE\insertsectionhead\\\end{centering}}

\frame[containsverbatim]{ \frametitle{Goals}
  \begin{itemize}
    \item  The module \emph{Advanced Programming Tools} will teach:
    \begin{itemize}
      \item binary utilities
      \item tools for performance analysis
      \item tools for debugging
    \end{itemize}
    \item We will use Cactus as an example of an Application Framework.
  \end{itemize}
}

\section{Binary Utilities}
\frame{\frametitle{}\begin{centering}\LARGE\insertsectionhead\\\end{centering}}

\frame[containsverbatim]{ \frametitle{Binary Utilities}
    Cactus normally shields programmers from the need
    to know about the details of the compilation process,
    but occasionally some difficult problem will be
    encountered that requires more analysis.
    \begin{itemize}
    \item nm/readelf
    \item c++filt
    \item ldd
    \end{itemize}
}

\subsection{nm - list symbols}
\frame{\frametitle{}\begin{centering}\LARGE\insertsubsectionhead\\\end{centering}}

\frame[containsverbatim]{ \frametitle{nm - list symbols}
  This tool is useful for figuring out dependencies when
  things are not compiling properly.
  \begin{itemize}
    \item T indicates a subroutine defined in the current object
    \item U indicates a subroutine or external variable that is
            defined in an external object file or library
    \item B a variable that is defined in the uninitialized data
            segment
    \item C a variable that is defined in the uninitialized data
            segment
    \item b a static variable that is defined in the uninitialized data
            segment
    \item D a variable that is defined in the initialized data
            segment
    \item d a variable that is defined in the initialized data
            segment
  \end{itemize}
}
\frame[containsverbatim]{ \frametitle{nm - list symbols}
    \begin{verbatim}
// demo.c
extern int var1,var2;
int var1;
int var2=3;

static int var3,var4=2;

void subroutine1();
void subroutine2();

void subroutine2() {
    var4=var3=1;
    subroutine1();
}
    \end{verbatim}
}

\frame[containsverbatim]{ \frametitle{nm - list symbols}
    \begin{verbatim}
[sbrandt@localhost]$ gcc -c demo.c
[sbrandt@localhost]$ nm demo.o
                 U subroutine1
0000000000000000 T subroutine2
0000000000000004 C var1
0000000000000000 D var2
0000000000000000 b var3
0000000000000004 d var4
    \end{verbatim}
}

\frame[containsverbatim]{ \frametitle{nm - list symbols}
Things turn out differently with the C++ compiler.
The C++ compiler scrambles type information into the
symbol name.
    \begin{verbatim}
[sbrandt@localhost]$ g++ -c demo.c
[sbrandt@localhost]$ nm demo.o
                 U _Z11subroutine1v
0000000000000000 T _Z11subroutine2v
0000000000000004 b _ZL4var3
0000000000000004 d _ZL4var4
                 U __gxx_personality_v0
0000000000000000 B var1
0000000000000000 D var2
    \end{verbatim}
}

\frame[containsverbatim]{ \frametitle{nm - list symbols}
    Note that -C unscrambles the C++ names as they appear
    inside the object files. This makes things more readable.
    \begin{verbatim}
[sbrandt@localhost]$ g++ -c demo.c
[sbrandt@localhost]$ nm -C demo.o
                 U subroutine1()
0000000000000000 T subroutine2()
0000000000000004 b var3
0000000000000004 d var4
                 U __gxx_personality_v0
0000000000000000 B var1
0000000000000000 D var2
    \end{verbatim}
}

\frame[containsverbatim]{ \frametitle{nm - list symbols}
    Using the -A flag to add the file name to each line
    {\small\begin{verbatim}
sbrandt@localhost$ gcc -c demo.c demo2.c
sbrandt@localhost$ nm -g -A *.o
demo2.o:0000000000000006 T main
demo2.o:                 U printf
demo2.o:0000000000000000 T subroutine1
demo2.o:                 U subroutine2
demo2.o:0000000000000004 C var1
demo.o:                 U subroutine1
demo.o:0000000000000000 T subroutine2
demo.o:0000000000000004 C var1
demo.o:0000000000000000 D var2
    \end{verbatim}
}
    Now you can combine with grep and see which object file
    defines a symbol and which object files need it.
    {\small\begin{verbatim}
sbrandt@localhost$ nm -g -A *.o|grep subroutine1
demo2.o:0000000000000000 T subroutine1
demo.o:                 U subroutine1
    \end{verbatim}
}
}

\frame[containsverbatim]{ \frametitle{nm - list symbols}
    Using the -u flag to list undefined symbols
    {\small\begin{verbatim}
sbrandt@localhost$ gcc -c -g demo.c demo2.c
sbrandt@localhost$ nm -uA *.o
demo2.o:                 U printf
demo2.o:                 U subroutine2
demo.o:                 U subroutine1
    \end{verbatim}
}
    If you'd like to know where you've tried to use those
    undefined symbols in your source code, use -l.
    {\small\begin{verbatim}
sbrandt@localhost$ nm -uAl *.o
demo2.o:                 U printf   /home/sbrandt/demo2.c:8
demo2.o:                 U subroutine2  /home/sbrandt/demo2.c:7
demo.o:                 U subroutine1   /home/sbrandt/demo.c:13
    \end{verbatim}
}
}

\subsection{readelf - display information about ELF tables}


\frame[containsverbatim]{ \frametitle{readelf - display information about ELF tables}
    ELF stands for ``Executable and Linking Format." ELF can be used to get
    essentially the same information as nm, with one important advantage. It
    can work on some files that nm can't (stripped shared libraries).
    Lines with UND aren't defined in the current source file.
    {\small\begin{verbatim}
sbrandt@localhost$ readelf -s demo.o

Symbol table '.symtab' contains 14 entries:
   Num:    Value          Size Type    Bind   Vis      Ndx Name
     0: 0000000000000000     0 NOTYPE  LOCAL  DEFAULT  UND 
     1: 0000000000000000     0 FILE    LOCAL  DEFAULT  ABS demo.c
     2: 0000000000000000     0 SECTION LOCAL  DEFAULT    1 
     3: 0000000000000000     0 SECTION LOCAL  DEFAULT    3 
     4: 0000000000000000     0 SECTION LOCAL  DEFAULT    4 
     5: 0000000000000000     4 OBJECT  LOCAL  DEFAULT    4 var3
     6: 0000000000000004     4 OBJECT  LOCAL  DEFAULT    3 var4
     7: 0000000000000000     0 SECTION LOCAL  DEFAULT    6 
    \end{verbatim}
}
}

\subsection{c++filt - demangle C++ and Java symbols}
\frame{\frametitle{}\begin{centering}\LARGE\insertsubsectionhead\\\end{centering}}

\frame[containsverbatim]{ \frametitle{c++filt - demangle C++ and Java symbols}
\begin{itemize}
\item Simple tool that transforms mangled C++ symbol definitions
      on the input stream to human readable form.
\item ``nm demo.o | c++filt" does the same thing as ``nm -C demo.o"
\item readelf has no -C flag, so ``readelf -s demo.o $|$ c++filt" is how to do it.
\item First version was written by yours truly in the 80's.
\end{itemize}
}

\subsection{ldd - shared library dependencies}
\frame{\frametitle{}\begin{centering}\LARGE\insertsubsectionhead\\\end{centering}}

\frame[containsverbatim]{ \frametitle{ldd - shared library dependencies}

    Ldd can tell you where your shared libraries are.
    \begin{verbatim}
[sbrandt@localhost]$ gcc --warn-all demo.c demo2.c
[sbrandt@localhost]$ ldd ./a.out
    linux-vdso.so.1 =>  (0x00007fff53dff000)
    libc.so.6 => /lib64/libc.so.6 (0x000000376d600000)
    /lib64/ld-linux-x86-64.so.2 (0x000000376d200000)

    \end{verbatim}
}
\frame[containsverbatim]{ \frametitle{ldd - shared library dependencies}
Sometimes ldd can't find your shared library. You can fix this by
modifying the LD\_LIBRARY\_PATH variable.

First, let's create our own shared library from our demo.c source
file. We need the -shared and -fPIC (Position Independent Code)
flags to make this work.
    \begin{verbatim}
[sbrandt@localhost]$ gcc -fPIC -shared -o libdemo.so demo.c
[sbrandt@localhost]$ gcc demo2.c libdemo.so
sbrandt@localhost$ ./a.out
./a.out: error while loading shared libraries: libdemo.so:
    cannot open shared object file: No such file or directory
    \end{verbatim}
}
\frame[containsverbatim]{ \frametitle{ldd - shared library dependencies}
What went wrong?
    \begin{verbatim}
[sbrandt@localhost]$ ldd a.out
    linux-vdso.so.1 =>  (0x00007ffffc3ff000)
    libdemo.so => not found
    libc.so.6 => /lib64/libc.so.6 (0x000000376d600000)
    /lib64/ld-linux-x86-64.so.2 (0x000000376d200000)
    \end{verbatim}
You see that libdemo.so is not found? That's because linux won't
look in the current directory by default. We have to tell it
to do so.
    \begin{verbatim}
[sbrandt@localhost]$ LD_LIBRARY_PATH=. ldd a.out
    linux-vdso.so.1 =>  (0x00007fff30784000)
    libdemo.so => ./libdemo.so (0x00007f9634fe7000)
    libc.so.6 => /lib64/libc.so.6 (0x000000376d600000)
    /lib64/ld-linux-x86-64.so.2 (0x000000376d200000)
    \end{verbatim}
}

\frame[containsverbatim]{ \frametitle{alternate shared library dependency tracing}
An alternative to using ldd is to set some variables. Note that LD\_DEBUG appends the process id to the file name. That's helpful for MPI. Note that "ls -t \verb=|= head -1" shows the most recently modified file.

    {\small\begin{verbatim}
[sbrandt@localhost]$ export LD_DEBUG=libs
[sbrandt@localhost]$ export LD_DEBUG_OUTPUT=debug.txt
[sbrandt@localhost]$ ./a.out
./a.out: error while loading shared libraries: libdemo.so:
    cannot open shared object file: No such file or directory
[sbrandt@localhost]$ ls -t | head -1
debug.txt.4478
[sbrandt@localhost]$ cat debug.txt.4478
      4478: find library=libdemo.so [0]; searching
      4478:  search cache=/etc/ld.so.cache
      4478:  search path=/lib64/tls/x86_64:/lib64/tls:/lib64/x86_64:/lib64:/usr/lib64/tls/x86_64:/usr/lib64/tls:/usr/lib64/x86_64:/usr/lib64        (system search path)
      4478:   trying file=/lib64/tls/x86_64/libdemo.so
      4478:   trying file=/lib64/tls/libdemo.so
      ...
    \end{verbatim}
}
}

\section{Debugging Utilities}
\frame{\frametitle{}\begin{centering}\LARGE\insertsectionhead\\\end{centering}}

\frame[containsverbatim]{ \frametitle{Debugging Utilities}
Memory problems (indexing outside of array bounds, using unallocated memory, etc.) and race conditions are insidious problems.
Results may be difficult to reproduce and the causes difficult to identify. You have to debug early and often if you want to stay sane.
These tools can help you.
\begin{itemize}
\item gdb
\item Valgrind
\item Helgrind
\item clang
\item misc
\end{itemize}
}
\subsection{gdb}
\frame{\frametitle{}\begin{centering}\LARGE\insertsubsectionhead\\\end{centering}}

\frame[containsverbatim]{ \frametitle{gdb}
The gnu debugger is widely available. Very helpful when you have a memory
problem. It has many useful features:
\begin{enumerate}
\item where - Gives the stack trace at the point of failure
\item print - Prints the contents of a variable
\item break/clear - Set/clear break point
\item step/next - advance line by line
\item run/cont - run/continue
\end{enumerate}
}
\frame[containsverbatim]{ \frametitle{gdb}
Let's see it in action.
This code has an obvious problem:
\begin{verbatim}
int main() {
    int *a = 0;
    a[10] = 4;
    return 0;
}
\end{verbatim}
}
\frame[containsverbatim]{ \frametitle{gdb}
\begin{verbatim}
sbrandt@sbrandt-think examples$ gdb ./test1
GNU gdb (GDB) Fedora (7.6-34.fc19)
Copyright (C) 2013 Free Software Foundation, Inc.
...
Reading symbols from /home/sbrandt/repos/sci-comp-2013/Module-C/examples/test1...done.
(gdb) run 
Starting program: /home/sbrandt/repos/sci-comp-2013/Module-C/examples/test1 

Program received signal SIGSEGV, Segmentation fault.
0x00000000004005c4 in main () at test1.c:4
4       a[10] = 4;
Missing separate debuginfos, use: debuginfo-install glibc-2.17-14.fc19.x86_64 libgcc-4.8.1-1.fc19.x86_64 libstdc++-4.8.1-1.fc19.x86_64
(gdb) where
#0  0x00000000004005c4 in main () at test1.c:4
(gdb) 
\end{verbatim}
}
\frame[containsverbatim]{ \frametitle{gdb}
\begin{verbatim}
#include <stdlib.h>
void set10(int *a) {
    a[10] = 4;
}
int main() {
    int *a = (int *)malloc(11*sizeof(int));
    set10(a);
    set10(0);
    return 0;
}

(gdb) where
#0  0x0000000000400600 in set10 (a=0x0) at test1b.c:3
#1  0x0000000000400637 in main () at test1b.c:8
\end{verbatim}
}
\frame[containsverbatim]{ \frametitle{gdb}
{\scriptsize
\begin{verbatim}
(gdb) run
(gdb) break test1b.c:6
Breakpoint 1, main () at test1b.c:6
6       int *a = (int *)malloc(11*sizeof(int));
(gdb) step
7       set10(a);
(gdb) step
set10 (a=0x602010) at test1b.c:3
3       a[10] = 4;
(gdb) step
4   }
(gdb) step
main () at test1b.c:8
8       set10(0);
(gdb) step
set10 (a=0x0) at test1b.c:3
3       a[10] = 4;
(gdb) step
Program received signal SIGSEGV, Segmentation fault.
0x0000000000400600 in set10 (a=0x0) at test1b.c:3
3       a[10] = 4;
\end{verbatim}
}
}

\subsection{Valgrind}
\frame{\frametitle{}\begin{centering}\LARGE\insertsubsectionhead\\\end{centering}}

\frame[containsverbatim]{ \frametitle{Valgrind}
Valgrind simulates your program in memory. It is slow, but it reveals
many problems.
\begin{verbatim}
[sbrandt@localhost t]$ cat test.c
#include <stdio.h>
int main() {
    int a;
    if(a < 0) printf("a is negative\n");
    return 0;
}
[sbrandt@localhost t]$ gcc -g test.c
[sbrandt@localhost t]$ valgrind ./a.out
...
==3512== Conditional jump or move depends on uninitialized value(s)
==3512==    at 0x4004D0: main (test.c:4)
...
\end{verbatim}
}

\frame[containsverbatim]{ \frametitle{Valgrind}
\begin{verbatim}
[sbrandt@localhost t]$ cat test2.c
#include <stdlib.h>
int main() {
    int *a = (int*)malloc(10*sizeof(int));
    a[10] = 4;
    return 0;
}
[sbrandt@localhost t]$ valgrind ./a.out
...
==3482== Invalid write of size 4
==3482==    at 0x4004E2: main (in /home/sbrandt/c/t/a.out)
==3482==  Address 0x4c0c068 is 0 bytes after a block of size 40 alloc'd
==3482==    at 0x4A0515D: malloc (vg_replace_malloc.c:195)
==3482==    by 0x4004D5: main (in /home/sbrandt/c/t/a.out)
...
\end{verbatim}
}
\frame[containsverbatim]{ \frametitle{Valgrind}
We fix test2.c like this:
\begin{verbatim}
#include <stdlib.h>
int main() {
    int *a = (int*)malloc(10*sizeof(int));
    a[9] = 4;
    return 0;
}
\end{verbatim}
Can Valgrind tell us anything more about our program?
}
\frame[containsverbatim]{ \frametitle{Valgrind}
{\scriptsize
\begin{verbatim}
sbrandt@sbrandt-think examples$ valgrind --leak-check=full ./a.out
==15928== Memcheck, a memory error detector
==15928== Copyright (C) 2002-2012, and GNU GPL'd, by Julian Seward et al.
==15928== Command: ./a.out
==15928== 
==15928== HEAP SUMMARY:
==15928==     in use at exit: 40 bytes in 1 blocks
==15928==   total heap usage: 1 allocs, 0 frees, 40 bytes allocated
==15928== 
==15928== 40 bytes in 1 blocks are definitely lost in loss record 1 of 1
==15928==    at 0x4A06409: malloc (in /usr/lib64/valgrind/vgpreload_memcheck-amd64-linux.so)
==15928==    by 0x400541: main (in /home/sbrandt/repos/sci-comp-2013/Module-C/examples/a.out)
==15928== 
==15928== LEAK SUMMARY:
==15928==    definitely lost: 40 bytes in 1 blocks
==15928==    indirectly lost: 0 bytes in 0 blocks
==15928==      possibly lost: 0 bytes in 0 blocks
==15928==    still reachable: 0 bytes in 0 blocks
==15928==         suppressed: 0 bytes in 0 blocks
==15928== 
==15928== For counts of detected and suppressed errors, rerun with: -v
==15928== ERROR SUMMARY: 1 errors from 1 contexts (suppressed: 2 from 2)
\end{verbatim}
}
}

\subsection{Helgrind}
\frame{\frametitle{}\begin{centering}\LARGE\insertsubsectionhead\\\end{centering}}

\frame[containsverbatim]{ \frametitle{Helgrind}
Useful for debugging threads. Prone to false positives.
{\small\begin{verbatim}
[sbrandt@localhost]$ cat race.c
#include <stdio.h>
#include <pthread.h>
int i = 0;
void *task(void *v) {
    for(int j=0;j<100000;j++) i++;
    return 0;
}
int main() {
    const int N = 10; pthread_t p[N];
    for(int i=0;i<N;i++)
        pthread_create(&p[i],0,task,0);
    for(int i=0;i<N;i++)
        pthread_join(p[i],0);
    printf("i=%d\n",i);
}
\end{verbatim}
}
}

\frame[containsverbatim]{ \frametitle{Helgrind}
We can see that we have a problem on line 5.
{\small\begin{verbatim}
[sbrandt@localhost]$ gcc -std=c99 -g race.c -lpthread
[sbrandt@localhost]$ valgrind --tool=helgrind \
    --log-file=val.out ./a.out 2>&1 | grep race.c
i=1000000
sbrandt@localhost$ grep race.c val.out|sort -u
==20857==    at 0x4005D5: task (race.c:5)
==20857==    at 0x4005DE: task (race.c:5)
==20857==    by 0x40067A: main (race.c:11)
\end{verbatim}
}
}

\frame[containsverbatim]{ \frametitle{Helgrind}
Here's one way we can fix it, using an atomic operation.
{\small\begin{verbatim}
#include <stdio.h>
#include <pthread.h>
int i = 0;
void *task(void *v) {
    for(int j=0;j<100000;j++)
        __sync_add_and_fetch(&i,1);
    return 0;
}
int main() {
    const int N = 10; pthread_t p[N];
    for(int i=0;i<N;i++)
        pthread_create(&p[i],0,task,0);
    for(int i=0;i<N;i++)
        pthread_join(p[i],0);
    printf("i=%d\n",i);
}
\end{verbatim}
}
}

\frame[containsverbatim]{ \frametitle{Helgrind}
Now everything is fixed!
{\small\begin{verbatim}
[sbrandt@localhost]$ gcc -std=c99 -g race.c -lpthread
[sbrandt@localhost]$ valgrind --tool=helgrind \
    --log-file=val.out ./a.out 2>&1 | grep race.c
i=1000000
sbrandt@localhost$ grep race.c val.out|sort -u
\end{verbatim}
}
Note that Helgrind works well with pthreads, but
doesn't work well with other forms of thread
parallelism.
}

%\subsection{Electric Fence}
%\frame{\frametitle{}\begin{centering}\LARGE\insertsubsectionhead\\\end{centering}}

%\frame[containsverbatim]{ \frametitle{Electric Fence}
%Electric Fence is a compile time tool that can help you
%spot a number of errors during runtime. Electric Fence
%works by replacing malloc.
%\begin{verbatim}
%[sbrandt@localhost t]$ cat test2.c
%#include <stdlib.h>
%int main() {
%    int *a = (int*)malloc(10*sizeof(int));
%    a[10] = 4;
%    return 0;
%}
%[sbrandt@localhost t]$ gcc -g test2.c -lefence
%\end{verbatim}
%}

%\frame[containsverbatim]{ \frametitle{Electric Fence}
%Electric fence triggers an abort when an error occurs. You can then
%use gdb to locate the line on which the error occurred.
%\begin{verbatim}
%[sbrandt@localhost t]$ gdb ./a.out
%GNU gdb (GDB) Fedora (7.1-34.fc13)
%(gdb) run 
%Starting program: /home/sbrandt/c/t/a.out 

%  Electric Fence 2.2.2 Copyright (C) 1987-1999 Bruce Perens <bruce@perens.com>

%Program received signal SIGSEGV, Segmentation fault.
%0x00000000004005e2 in main () at test2.c:4
%4       a[10] = 4;
%Missing separate debuginfos, use: debuginfo-install ElectricFence-2.2.2-27.fc12.x86_64 glibc-2.12.2-1.x86_64
%\end{verbatim}
%}

%\frame[containsverbatim]{ \frametitle{Electric Fence}
%\begin{itemize}
%\item Electric Fence dies when malloc(0) is called.
%\item malloc(0) is called in the regular expression library
%  by Cactus. Electric Fence will ignore this error if you set
%  the environment variable EF\_ALLOW\_MALLOC\_0=1.
%\item Electric Fence greatly increases the amount of memory used
%  by an application. In Cactus, this prevents it from being a useful
%  diagnostic.
%\end{itemize}
%}

\subsection{Clang}
\frame{\frametitle{}\begin{centering}\LARGE\insertsubsectionhead\\\end{centering}}

\frame[containsverbatim]{ \frametitle{clang}
\begin{itemize}
\item clang stands for C-language compiler. Based on the llvm (Low Level virtual machine) project.
\item download it here \verb=http://clang.llvm.org/get_started.html#build=
\item performs details source code analysis
\end{itemize}
}

\frame[containsverbatim]{ \frametitle{clang}
Just using the clang compiler can help you find new bugs
\begin{verbatim}
[sbrandt@localhost t]$ cat test3.c
int main() {
    int b[1];
    b[2] = 3;
}
[sbrandt@localhost t]$ clang test3.c
test3.c:3:5: warning: array index of '2' indexes past the end of an array
      (that contains 1 element) [-Warray-bounds]
    b[2] = 3;
    ^ ~
test3.c:2:5: note: array 'b' declared here
    int b[1];
    ^
1 warning generated.
\end{verbatim}
}

\frame[containsverbatim]{ \frametitle{clang}
Another way to discover problems is with -fsanitize=undefined-trap -fsanitize-undefined-trap-on-error.
{\small\begin{verbatim}
[sbrandt@localhost t]$ cat test4.c
void foo(int n,int v) {
    int f[2];
    f[n] = v;
}
int main() {
    foo(3,3);
    return 0;
}
[sbrandt@localhost t]$ clang -g -fsanitize=undefined-trap \
    -fsanitize-undefined-trap-on-error test4.c
[sbrandt@localhost t]$ gdb -eval-command=run ./a.out
Program received signal SIGILL, Illegal instruction.
0x000000000040049c in foo (n=3, v=3) at test4.c:3
3       f[n] = v;
\end{verbatim}
}
}

\frame[containsverbatim]{ \frametitle{clang}
Integer overflow problems can be discovered with -ftrapv.
\begin{verbatim}
[sbrandt@localhost t]$ cat test5.c
#include <limits.h>
#include <stdio.h>
void foo(int n) {
    if(n > n+1) printf("overflow\n");
}
int main() {
    foo(INT_MAX);
}
[sbrandt@localhost t]$ clang -g -ftrapv test5.c
[sbrandt@localhost t]$ gdb -eval-command=run ./a.out
Program received signal SIGILL, Illegal instruction.
0x00000000004004f3 in foo (n=2147483647) at test5.c:4
4       if(n > n+1) printf("overflow\n");
\end{verbatim}
}

\subsection{Misc.}
\frame{\frametitle{}\begin{centering}\LARGE\insertsubsectionhead\\\end{centering}}

\frame[containsverbatim]{ \frametitle{Misc.}
\begin{itemize}
\item gfortran -fbounds-check - generates bounds checking code for
fortran. This flag is {\it silently ignored} for gcc and g++.
\item Cactus debug build - automatically detects indexes out of
bounds in grid functions, plus does other things
\item In general, using a different compiler or architecture
can be helpful in exposing bugs or problems in code.
\item \verb=#include <assert.h>= is very helpful. Use -DNDEBUG
to make assertions go away.
\end{itemize}
}

\frame[containsverbatim]{ \frametitle{Misc.}
\begin{verbatim}
[sbrandt@localhost ~]$ cat here.h
#ifndef HERE_H_
#define HERE_H_
#include <stdio.h>
#define HERE printf("%s,%d\n",__FILE__,__LINE__);
#define VARN(X) printf("%s=%d\n",#X,X);
#define VAR(X) printf("%s=%d ",#X,X);
#endif
\end{verbatim}
}

\section{Performance Utilities}
\frame{\frametitle{}\begin{centering}\LARGE\insertsectionhead\\\end{centering}}

\frame[containsverbatim]{ \frametitle{Performance Utilities}
\begin{itemize}
\item gprof
\item Coverage Testing
\item Memusage
\item Profile-Guided Optimization
\item hpctoolkit
\item perfexpert
\end{itemize}
}

\subsection{Gprof}
\frame{\frametitle{}\begin{centering}\LARGE\insertsubsectionhead\\\end{centering}}

\frame[containsverbatim]{ \frametitle{gprof}
Gprof provides basic, low-level profiling of source code. Helps you find your performance
hot spots and target your optimization efforts. To use, build your code with \verb=-pg=.

\begin{verbatim}
g++ -pg -g -o wave wave.cpp
\end{verbatim}

When you run your code, a file named \verb=gmon.out= is generated.
To view the results, use the \verb=gprof= command.

\begin{verbatim}
gprof wave gmon.out -p
\end{verbatim}
}
\frame[containsverbatim]{ \frametitle{gprof}
{\scriptsize
\begin{verbatim}
sbrandt@sbrandt-think examples$ gprof wave gmon.out -p
Flat profile:

Each sample counts as 0.01 seconds.
  %   cumulative   self              self     total           
 time   seconds   seconds    calls  us/call  us/call  name    
100.27      0.01     0.01     4000     2.51     2.51  deriv(double*, double*)
  0.00      0.01     0.00    19702     0.00     0.00  x(int)
  0.00      0.01     0.00     8000     0.00     0.00  bound(double*)
  0.00      0.01     0.00     1000     0.00    10.03  update(int)
  0.00      0.01     0.00      200     0.00     0.00  print_minmax(int)
  0.00      0.01     0.00      200     0.00     0.00  print(double*, int)
  0.00      0.01     0.00      100     0.00     0.00  sq(double)
  0.00      0.01     0.00        1     0.00     0.00  _GLOBAL__sub_I_u
  0.00      0.01     0.00        1     0.00     0.00  __static_initialization_and_destruction_0(int, int)
  0.00      0.01     0.00        1     0.00     0.00  init()

\end{verbatim}
}
Who knew that \verb=deriv= was our most expensive routine?
}

\subsection{Coverage Testing}
\frame{\frametitle{}\begin{centering}\LARGE\insertsubsectionhead\\\end{centering}}

\frame[containsverbatim]{ \frametitle{Coverage Testing}
Coverage testing builds in automatic instrumentation of your
code to determine which lines are being run and how often. It
can help you construct more thorough tests.
{\small\begin{verbatim}
[sbrandt@localhost t]$ gcc -ftest-coverage -fprofile-arcs test2.c
[sbrandt@localhost t]$ ./a.out
[sbrandt@localhost t]$ ls -t | head -3
test2.gcda
a.out
test2.gcno
\end{verbatim}
}
}
\frame[containsverbatim]{ \frametitle{Coverage Testing}
{\small\begin{verbatim}
[sbrandt@localhost t]$ gcov test2.gcda
File 'test2.c'
Lines executed:100.00% of 4
test2.c:creating 'test2.c.gcov'

[sbrandt@localhost t]$ cat test2.c.gcov 
        -:    0:Source:test2.c
        -:    0:Graph:test2.gcno
        -:    0:Data:test2.gcda
        -:    0:Runs:1
        -:    0:Programs:1
        -:    1:#include <stdlib.h>
        1:    2:int main() {
        1:    3:    int *a = (int*)malloc(10*sizeof(int));
        1:    4:    a[10] = 4;
        1:    5:    return 0;
        -:    6:}
\end{verbatim}
}
}

\subsection{Memusage}
\frame{\frametitle{}\begin{centering}\LARGE\insertsubsectionhead\\\end{centering}}

\frame[containsverbatim]{ \frametitle{Memusage}
Memusage is a utility which can help you monitor your
heap and stack usage.
{\small\begin{verbatim}
sbrandt@localhost$ cat fib.c
#include <stdio.h>

int fib(int n) {
    if(n < 2)
        return n;
    return fib(n-1)+fib(n-2);
}

int main() {
    int *n = new int(35);
    printf("fib(%d)=%d\n",*n,fib(*n));
    return 0;
}
sbrandt@localhost$ g++ fib.c -lmemusage
\end{verbatim}
}
}
\frame[containsverbatim]{ \frametitle{Memusage}
{\small\begin{verbatim}
sbrandt@localhost$ MEMUSAGE_OUTPUT=mem.out \
    LD_PRELOAD=/usr/lib64/libmemusage.so ./a.out
fib(35)=9227465

Memory usage summary: heap total: 4, heap peak: 4, stack peak: 2976
         total calls   total memory   failed calls
 malloc|          1              4              0
realloc|          0              0              0  (nomove:0, dec:0, free:0)
 calloc|          0              0              0
   free|          2              0
Histogram for block sizes:
    0-15              1 100% ==================================================
\end{verbatim}
}
}

\frame[containsverbatim]{ \frametitle{Memusage}
{\small\begin{verbatim}
sbrandt@localhost$ memusagestat -o mem.png mem.out -x 500 -y 300
\end{verbatim}
}
\begin{figure}
\includegraphics[width=9cm]{pics/mem}
\end{figure}
}

\subsection{Profile-Guided Optimization}
\frame{\frametitle{}\begin{centering}\LARGE\insertsubsectionhead\\\end{centering}}

\frame[containsverbatim]{ \frametitle{Profile Guided Optimization}
\begin{itemize}
\item Run your code and see how fast it is
\item Compile with -fprofile-generate
\item Run your code (it will be slightly slower)
\item Compile again with -fprofile-use
\item Run your code again, and see how much faster it is
\item Works well if you have lots of if's
\end{itemize}
}

\frame[containsverbatim]{ \frametitle{Profile Guided Optimization}
{\small\begin{verbatim}
#include <stdlib.h>
void bubble(int *a,int n) {
    for(int j=0;j<n-1;j++)
        for(int i=0;i<n-1;i++)
            if(a[i] > a[i+1]) {
                int s = a[i];
                a[i] = a[i+1];
                a[i+1] = s;
            }
}
int main() {
    const int n = 60000;
    int *a = new int[n];
    for(int i=0;i<n;i++)
        a[i] = rand();
    bubble(a,n);
    return 0;
}
\end{verbatim}
}
}

\frame[containsverbatim]{ \frametitle{Profile Guided Optimization}
\begin{itemize}
\item Speedup is approximately 6\% for Bubble Sort on mike (6.44 to 6.08)
\item Can be used to speed up Firefox and gcc
\item Used extensively by the Java Virtual Machine
\item With the Intel compilers there's -prof-gen and -prof-use
\item Actually slows down!
\end{itemize}
}

%\subsection{oprofile}
%\frame{\frametitle{}\begin{centering}\LARGE\insertsubsectionhead\\\end{centering}}


%\frame[containsverbatim]{ \frametitle{oprofile}
%    \begin{itemize}
%    \item PAPI is not available on all machines (i.e. not on LONI)
%    \item Hardware counters available through OProfile
%    \item OProfile requires root
%    \item OProfile can be used on sudo is available on LONI
%    \end{itemize}
%}

%\frame[containsverbatim]{ \frametitle{oprofile}
%    You can find out what your privileges are using
%    sudo -l.
%    \begin{verbatim}
%User sbrandt may run the following commands on this host:
%    (gold) NOPASSWD: /usr/local/packages/gold/bin/glsuser,
%    /usr/local/packages/gold/bin/glsproject,
%    /usr/local/packages/gold/bin/glsalloc,
%    /usr/local/packages/gold/bin/bal.pl,
%    /usr/local/packages/gold/bin/gbalance
%    (root) NOPASSWD: /usr/local/bin/opsafe
%    \end{verbatim}
%}


%\frame[containsverbatim]{ \frametitle{oprofile}
%\begin{itemize}
%\item If you want to profile applications on Queen Bee, you can't run
%the opcontrol command directly. Instead, you can run a script called
%opsafe that limits what the command can do.

%\item If you are running in the batch queue, you will need to put
%``{\tt pbsdsh -u}'' before each opsafe command. This will make sure that
%oprofile is running on all the nodes you are using in your computation.

%\item Because the opsafe command runs as root, you'll need to precede it
%with ``{\tt sudo -u root}''.

%\item In the next slide, ``{\tt opsafe}'' should be taken as an
%abbreviation for ``{\tt pbsdsh -u sudo /usr/local/bin/opsafe}''.
%\end{itemize}
%}
%\frame[containsverbatim]{ \frametitle{oprofile}
%\begin{itemize}
%\item {\tt opsafe --shutdown}
%Makes sure oprofile is turned off before we configure and start it.
%\item {\tt opsafe --setup --no-vmlinux --event=CPU\_CLK\_UNHALTED:400000}
%\item {\tt \verb=opsafe --start=}
%\item {\it ./run\_your\_code}
%\item {\tt \verb=opsafe --dump=}
%\item {\tt opannotate --source} {\it full\_path/run\_your\_code} {\tt --output-dir=}{\it full\_path}{\tt\verb=/anno$HOSTNAME=}
%   This command actually needs to be run by ``{\tt pbsdsh}'' from inside a script. The script allows each the \verb=$HOSTNAME= variable to expand differently.
%\end{itemize}
%}

\subsection{HPC Toolkit}
\frame{\frametitle{}\begin{centering}\LARGE\insertsubsectionhead\\\end{centering}}

\frame[containsverbatim]{ \frametitle{HPC Toolkit}
To demonstrate we will use the HPC Toolkit on Arete (arete.cct.lsu.edu).
We'll run Cactus for 200 steps, then run the viewer.
\begin{itemize}
\item Compile with debug mode:
{\tiny\verb=sim build --debug --thornlist ../WaveDemo.th --optionlist local.cfg=}
\item Run with hpcrun:
{\tiny\verb=mpiexec -np 2 hpcrun -e PAPI_TOT_CYC:1000000 -e PAPI_L2_DCM:100000 \=
\verb=-e PAPI_FP_OPS:500000 -e PAPI_TLB_DM:100000 ./exe/cactus_sim-debug ./pars/WaveDemo.par=}
\item Postprocess with hpcstruct:
{\tiny \verb=hpcstruct ./exe/cactus_sim-debug=}
\item Postprocess with hpcprof:
{\tiny \verb=hpcprof -S ./cactus_sim-debug.hpcstruct -I ./ hpctoolkit-cactus_sim-debug-measurements/=}
\item View:
\tiny{\verb=hpcviewer hpctoolkit-cactus_sim-debug-database/=}
\end{itemize}
}

\frame[containsverbatim]{ \frametitle{HPC Toolkit}
    \begin{center}
    \includegraphics[width=9cm]{pics/hpcviewer}\mbox{\hspace{0.5em}}
    \end{center}
}

\subsection{Perfexpert}
\frame{\frametitle{}\begin{centering}\LARGE\insertsubsectionhead\\\end{centering}}

\frame[containsverbatim]{ \frametitle{perfexpert}
This is a tool from TACC that attempts to turn the output of HPC Toolkit
into something easier to understand.
\begin{itemize}
\item Home Page: http://www.tacc.utexas.edu/perfexpert/
\item Automatically runs a sequence of experiments
\item Shows information about the most significant sections of code
      and rate loops on how well they perform.
\item AutoSCOPE attempts to give concrete advice on how to improve
      problem areas of code.
\end{itemize}
}

\frame[containsverbatim]{ \frametitle{perfexpert}
Running Perfexpert has fewer steps than running HPC Toolkit. First
you run experiments, then you run the filter on the output (experiment.xml).
When you run the regular perfexpert tool, you specify what level of
analysis you need. Below we choose to ignore diagnostics for loops that
take less than 10\% of the run time.
\begin{itemize}
\item {\small\verb=[sbrandt@master Cactus]$ perfexpert_run_exp= 
  \verb=./exe/cactus_sim ./pars/WaveDemo.par=}
\item {\small\verb=[sbrandt@master Cactus]$ perfexpert 0.1= \verb=./experiment.xml > perfout.txt=}
\end{itemize}
}

\frame[containsverbatim]{ \frametitle{perfexpert}
{\tiny
\begin{verbatim}
Total running time for "./experiment.xml" is 25.236 sec

Function WaveToyC_Evolution() at WaveToy.c:36 (41.7% of the total runtime)
===============================================================================
ratio to total instrns       %  0.........25...........50.........75........100
   - floating point      :   22 **********
   - data accesses       :   24 ***********
* GFLOPS (% max)         :   10 *****
-------------------------------------------------------------------------------
performance assessment     LCPI good......okay......fair......poor......bad....
* overall                :  0.6 >>>>>>>>>>>
upper bound estimates
* data accesses          :  1.1 >>>>>>>>>>>>>>>>>>>>>>>
   - L1d hits            :  0.9 >>>>>>>>>>>>>>>>>>
   - L2d hits            :  0.1 >>
   - L2d misses          :  0.2 >>>
* instruction accesses   :  0.6 >>>>>>>>>>>
   - L1i hits            :  0.6 >>>>>>>>>>>
   - L2i hits            :  0.0 >
   - L2i misses          :  0.0 >
* data TLB               :  0.0 >
* instruction TLB        :  0.0 >
* branch instructions    :  0.0 >
   - correctly predicted :  0.0 >
   - mispredicted        :  0.0 >
* floating-point instr   :  0.6 >>>>>>>>>>>>>
   - fast FP instr       :  0.6 >>>>>>>>>>>>>
   - slow FP instr       :  0.0 >

Loop in function WaveToyC_Evolution() at WaveToy.c:73 (41.4% of the total runtime)
===============================================================================
\end{verbatim}
}
}

\frame[containsverbatim]{ \frametitle{perfexpert}
    \begin{center}
    \includegraphics[width=10cm]{pics/AutoSCOPE}\mbox{\hspace{0.5em}}
    \end{center}
}

\frame[containsverbatim]{ \frametitle{perfexpert}
    \begin{center}
    \includegraphics[width=10cm]{pics/AutoSCOPE2}\mbox{\hspace{0.5em}}
    \end{center}
}
\subsection{Homework}

\frame[containsverbatim]{
    \frametitle{Homework}
    The following questions are based on the wave.cpp and qsort.c programs
    in the public SVN repository.
    \begin{enumerate}
    \item Debug wave.cpp using Valgrind. What problems did you find?
    \item Use profile guided optimization to speed up qsort.c. What speedup do you get?
    \end{enumerate}
}

\end{document}
