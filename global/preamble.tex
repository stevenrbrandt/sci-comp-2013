\documentclass{beamer}

\usepackage[utf8]{inputenc}
%\usepackage{beamerthemesplit}
\usepackage{url}
\usepackage{tikz}
\usepackage{alltt}
\usepackage{listings}
\usepackage{marvosym}
\usepackage{color}
\usepackage[multidot]{grffile}
\usepackage{multirow}
\usepackage{array}
\usepackage{setspace}
\usepackage{hyperref}
\usepackage{verbatim}
\usepackage{fancyvrb}
%\hypersetup{colorlinks=true, linkcolor=blue,  anchorcolor=blue,  
%citecolor=blue, filecolor=blue, menucolor=blue, pagecolor=blue,  
%urlcolor=blue} 
\lstset{keywordstyle=\bfseries\color{brown},
        stringstyle=\ttfamily,
        commentstyle=\color{blue}\textit,
        showstringspaces=false}

\useoutertheme{}
\usetheme{Madrid}
\graphicspath{{pics/}{global/}
{pics/I/}{pics/A1/}{pics/A2/}{pics/A3/}{pics/A4/}{pics/A5/}{pics/A6/}{pics/A7/}
}

\logo{\includegraphics[height=1cm]{ProcessHorizontal}} 

\institute{Center for Computation and Technology\\Louisiana State University, Baton Rouge, LA}

\setbeamertemplate{navigation symbols}{} 

\title{CSC 7700: Scientific Computing}

% We want to use the infolines outer theme because it does not use a lot of
% space, but it also tries to print an institution and the slide
% numbers (which we might not want to show). Therefore, we here redefine the
% footline ourselfes - mostly a copy & paste from
% /usr/share/texmf/tex/latex/beamer/themes/outer/beamerouterthemeinfolines.sty
\defbeamertemplate*{footline}{infolines theme without institution and slide numbers}
{
  \leavevmode%
  \hbox{%
  \begin{beamercolorbox}[wd=.25\paperwidth,ht=2.25ex,dp=1ex,center]{author in head/foot}%
    \usebeamerfont{author in head/foot}\insertshortauthor
  \end{beamercolorbox}%
  \begin{beamercolorbox}[wd=.5\paperwidth,ht=2.25ex,dp=1ex,center]{title in head/foot}%
    \usebeamerfont{title in head/foot}\insertshorttitle
  \end{beamercolorbox}%
  \begin{beamercolorbox}[wd=.25\paperwidth,ht=2.25ex,dp=1ex,center]{date in head/foot}%
    \usebeamerfont{date in head/foot}\insertshortdate{}
  \end{beamercolorbox}}%
  \vskip0pt%
}

% Some useful commands
\newcommand{\abspic}[4]
 {\vspace{ #2\paperheight}\hspace{ #3\paperwidth}\includegraphics[height=#4\paperheight]{#1}\\
  \vspace{-#2\paperheight}\vspace{-#4\paperheight}\vspace{-0.0038\paperheight}}

\newcommand{\picw}[4]{{
 \usebackgroundtemplate{
 \color{black}\vrule width\paperwidth height\paperheight\hspace{-\paperwidth}\hspace{-0.01\paperwidth}
 \hspace{#4\paperwidth}\includegraphics[width=#3\paperwidth, height=\paperheight]{#1}}\logo{}
 \frame[plain]{\frametitle{#2}}
}}
\newcommand{\pic}[2]{\picw{#1}{#2}{}{0}}

\newcommand{\question}[1]{\frame{\frametitle{#1}
 \begin{centering}\Huge #1\\\end{centering}
}}

