\documentclass{beamer}

\usepackage[utf8]{inputenc}
%\usepackage{beamerthemesplit}
\usepackage{url}
\usepackage{tikz}
\usepackage{alltt}
\usepackage{listings}
\usepackage{marvosym}
\usepackage{color}
\usepackage[multidot]{grffile}
\usepackage{multirow}
\usepackage{array}
\usepackage{setspace}
\usepackage{hyperref}
\usepackage{verbatim}
\usepackage{fancyvrb}
%\hypersetup{colorlinks=true, linkcolor=blue,  anchorcolor=blue,  
%citecolor=blue, filecolor=blue, menucolor=blue, pagecolor=blue,  
%urlcolor=blue} 
\lstset{keywordstyle=\bfseries\color{brown},
        stringstyle=\ttfamily,
        commentstyle=\color{blue}\textit,
        showstringspaces=false}

\useoutertheme{}
\usetheme{Madrid}
\graphicspath{{pics/}{global/}
{pics/I/}{pics/A1/}{pics/A2/}{pics/A3/}{pics/A4/}{pics/A5/}{pics/A6/}{pics/A7/}
}

\logo{\includegraphics[height=1cm]{ProcessHorizontal}} 

\institute{Center for Computation and Technology\\Louisiana State University, Baton Rouge, LA}

\setbeamertemplate{navigation symbols}{} 

\title{CSC 7700: Scientific Computing}

% We want to use the infolines outer theme because it does not use a lot of
% space, but it also tries to print an institution and the slide
% numbers (which we might not want to show). Therefore, we here redefine the
% footline ourselfes - mostly a copy & paste from
% /usr/share/texmf/tex/latex/beamer/themes/outer/beamerouterthemeinfolines.sty
\defbeamertemplate*{footline}{infolines theme without institution and slide numbers}
{
  \leavevmode%
  \hbox{%
  \begin{beamercolorbox}[wd=.25\paperwidth,ht=2.25ex,dp=1ex,center]{author in head/foot}%
    \usebeamerfont{author in head/foot}\insertshortauthor
  \end{beamercolorbox}%
  \begin{beamercolorbox}[wd=.5\paperwidth,ht=2.25ex,dp=1ex,center]{title in head/foot}%
    \usebeamerfont{title in head/foot}\insertshorttitle
  \end{beamercolorbox}%
  \begin{beamercolorbox}[wd=.25\paperwidth,ht=2.25ex,dp=1ex,center]{date in head/foot}%
    \usebeamerfont{date in head/foot}\insertshortdate{}
  \end{beamercolorbox}}%
  \vskip0pt%
}

% Some useful commands
\newcommand{\abspic}[4]
 {\vspace{ #2\paperheight}\hspace{ #3\paperwidth}\includegraphics[height=#4\paperheight]{#1}\\
  \vspace{-#2\paperheight}\vspace{-#4\paperheight}\vspace{-0.0038\paperheight}}

\newcommand{\picw}[4]{{
 \usebackgroundtemplate{
 \color{black}\vrule width\paperwidth height\paperheight\hspace{-\paperwidth}\hspace{-0.01\paperwidth}
 \hspace{#4\paperwidth}\includegraphics[width=#3\paperwidth, height=\paperheight]{#1}}\logo{}
 \frame[plain]{\frametitle{#2}}
}}
\newcommand{\pic}[2]{\picw{#1}{#2}{}{0}}

\newcommand{\question}[1]{\frame{\frametitle{#1}
 \begin{centering}\Huge #1\\\end{centering}
}}



%\usecolortheme[RGB={200,200,200}]{structure}

\subtitle[Module D]{{\large Module D: Simulations and Application Frameworks}\\*[0.3em]Lecture 1: Simulation Basics}
\author[Peter Diener]{Dr Peter Diener}
\date{November 15, 2013}
\usecolortheme[RGB={160,45,140}]{structure}

\begin{document}

\frame{\titlepage}

\section*{Outline}
\frame{\tableofcontents}

\section{Goals}
\frame{\frametitle{}\begin{centering}\LARGE\insertsectionhead\\\end{centering}}

\frame[containsverbatim]{ \frametitle{Goals}
  \begin{itemize}
    \item  The module \emph{Simulations and Application Frameworks} will teach:
    \begin{itemize}
      \item what a simulation is,
      \item how a typical simulation code looks like,
      \item how it is used in practice,
      \item and what some of the major concerns in such a code are.
    \end{itemize}
    \item We will use Cactus as an example of an Application Framework.
  \end{itemize}
}

\section{Simulations}
\frame{\frametitle{}\begin{centering}\LARGE\insertsectionhead\\\end{centering}}

\frame[containsverbatim]{ \frametitle{What is a simulation?}
  \begin{center}
    \vspace{-9.0em}
    \includegraphics[width=6cm]{pics/car-crash}\mbox{\hspace{0.5em}}
    \raisebox{3.8em}{\includegraphics[width=4cm]{pics/black-holes}} 

    \raisebox{-3em}[0em][0em]{\includegraphics[width=11cm]{pics/simulations1}}
  \end{center}
}

\frame[containsverbatim]{ \frametitle{What is a simulation?}
  \begin{itemize}
    \item Flame propagation in combustion engine: \emph{understand} behavior
          that is too fast or too small.
    \item Hurricane modeling: \emph{predict} behavior.
    \item Car crash testing: \emph{engineer} better devices.
    \item Video games: \emph{create} a fantasy world.
    \item Black hole collisions: \emph{test} the basic laws of nature.
  \end{itemize}
}

\frame[containsverbatim]{ \frametitle{What is a simulation?}
  \begin{center}
    \includegraphics[height=7cm]{simulation1}
  \end{center}
}

\frame[containsverbatim]{ \frametitle{What is a simulation?}
  \begin{center}
    \includegraphics[height=7.5cm]{simulation2}
  \end{center}
}

\frame[containsverbatim]{ \frametitle{Ingredients of a simulation}
  \begin{itemize}
    \item From the laws of nature derive a mathematical model relating the
          variables describing the system you want to simulate. This often
          involves approximations.
    \item Decide on a discretisation scheme to use for your variables and your
          mathematical equations (more later).
    \item Write a computer code to solve the discretised equations.
    \item Setup initial data and run the simulation.
    \item Analyze and visualize the results.
  \end{itemize}
}

\frame[containsverbatim]{ \frametitle{Example: The wave equation}
Consider a fluid in 1D described at time $t$ and position $x$ by the density
$\rho$, pressure $p$ and velocity $u$.

From Newton's first law ($m a = F$) we have
\[
\rho\frac{\partial u}{\partial t} = -\frac{\partial P}{\partial x}.
\]
On the other hand if the velocity $u$ varies with position $x$ the pressure
will change
\[
\frac{\partial P}{\partial t} = -K \frac{\partial u}{\partial x},
\]
where $K$ is the incompressibility of the fluid.

Assuming that $K$ and $\rho$ does not vary with position we can combine the
equations into one equation for the pressure $P$
\[
\frac{\partial^2 P}{\partial t^2}=\frac{K}{\rho}\frac{\partial^2 P}{\partial x^2}.
\]
}

\frame[containsverbatim]{ \frametitle{The Initial Value Problem}
  The wave equation is a very simple example of a very large class of
  computational problems encountered in physics, chemistry and biology: The
  Initial Value Problem (IVP).
  \begin{itemize}
    \item Typically expressed in terms of a set of PDEs (partial differential
          equations) that tells us how a system is changing given a known state
          of the system.
    \item Starting from a set of \emph{Initial Conditions} we can then simulate the
          behavior of the system by evolving from one state to another in
          finite timesteps.
  \end{itemize}
}

\frame[containsverbatim]{ \frametitle{Discretisation}
  \begin{itemize}
    \item It would require an infinite amount of information to describe the
          complete state of continuum system (air, water, car body, etc.\ ).
    \item Instead we select a finite set of discrete points where we assign
          values to the field variables in order to reduce the number of
          degrees of freedom.
    \item There a many different ways to do this
      \begin{itemize}
        \item Finite differences (sample solution on a regular grid)
        \item Finite elements (small rigid triangles or tetrahedra)
        \item Finite volumes (small squares or cubes)
        \item Expansion of fields in a finite number of basis functions.
        \item Particles
      \end{itemize}
  \end{itemize}
}

\frame[containsverbatim]{ \frametitle{Discretisation}
  \begin{itemize}
    \item Using a finite (instead of infinite) number of degrees of freedom 
          introduces an approximation error.
    \item The accuracy of the approximation depends on the discretisation
          scheme and the resolution (how closely is the continuum sampled).
    \item For any discretisation scheme we can establish the order of
          accuracy. E.g. fourth order: $\epsilon(h) = O(h^4)$.
    \item Validation of simulation results requires careful convergence
          studies.
    \item The order of accuracy only gives a reliable measure of the error
          when the resolution is high enough (the convergent regime).
  \end{itemize}
}

\frame[containsverbatim]{ \frametitle{Discretisation example: The wave equation}
  Approximating the pressure $P(x,t)$ with a grid function $P_i^{(n)}$.
  \begin{minipage}[b]{3.05cm}
    \includegraphics[width=3cm]{pics/discretisation}
  \end{minipage}
  \raisebox{8em}{
  \begin{minipage}[t]{8.5cm}
    \begin{eqnarray}
      \frac{\partial^2 P}{\partial t^2} & = & 
       v^2\frac{\partial^2 P}{\partial x^2} \nonumber \\
      & \Downarrow & \nonumber \\
      \frac{P_i^{(n+1)}-2 P_i^{(n)}+P_i^{(n-1)}}{\Delta t^2} & = &
      v^2\frac{P_{i+1}^{(n)}-2 P_i^{(n)}+P_{i-1}^{(n)}}{\Delta x^2} \nonumber
    \end{eqnarray}
  \end{minipage}}
  The error from this time and space discretisation is $O(\Delta x^2)$.

}
\frame[containsverbatim]{ \frametitle{Caveat}
  \begin{itemize}
    \item Some systems are not described by PDEs but other types of
          mathematical relations.
    \item Sometimes it is the end state of a system that is interesting (like
          an equilibrium configuration) and not how it got there.
    \item \begin{minipage}[t]{6cm}
            If the initial data is an approximation or a guess, the result of
            a simulation may not be reliable.
          \end{minipage}
          \raisebox{-4em}{
            \begin{minipage}[t]{4.5cm}
              \includegraphics[width=4cm]{pics/garbage}
            \end{minipage}
          }
  \end{itemize}
}

\frame[containsverbatim]{ \frametitle{Simulation Overview}
  \begin{itemize}
    \item Derive a mathematical model for the system you want to simulate.
    \item Discretise the resulting PDEs (or whatever other form they are in).
    \item Decide on the shape and size of the computational domain (normally
          it's impossible to simulate the whole universe).
    \item Decide on the resolution.
    \item Decide on how to handle the boundaries of the computational domain.
    \item Set up initial conditions.
    \item Evolve using many small timesteps.
    \item Output data along the way.
    \item Analyse and visualize the data.
    \item Write the paper.
  \end{itemize}
}

\frame[containsverbatim]{ \frametitle{Structure of a Simulation Code}
  \begin{itemize}
    \item Variables (solution) stored in large ``state vectors''
      \begin{itemize}
        \item There can be many (e.g. billions) elements in the state vector
        \item Best handled in efficient container structure like a Fortran 
              array, C or C++ vector or a tree structure.
        \item The code needs constructs to iterate over these elements
              or subsets of elements.
      \end{itemize}
    \item Routines to set up the initial condition.
    \item Routines to perform many identical time steps.
    \item I/O methods to write solution to disk and optionally checkpoint the
          simulation.
    \item Should be able to run as batch job without user intervention.
  \end{itemize}
}

\frame[containsverbatim]{ \frametitle{Initial Data}
  \begin{itemize}
    \item Analytical initial data can just be calculated when the
          simulation starts. 
    \item Numerical initial data may be solved internally at the beginning or
          be read in from file.
    \item Can be set up from reading in a checkpoint file (continuation run).
    \item Initial data can be large:\\
          1 billion elements,\\
          25 variables/element,\\
          8 Byte/variable: total 200 GByte.
  \end{itemize}
}

\frame[containsverbatim]{ \frametitle{Parallel Computing}
  \begin{itemize}
    \item Cannot store the solution on a single node; parallel programming
          using MPI is necessary.
    \item At the moment only Fortran, C and C++ are viable languages for
          programming a supercomputer.
    \item Research is underway in other (maybe simpler) ways like Unified
          Parallel C, Co-Array Fortran or HPX (here at LSU)
    \item Simulations take long:\\
          1 billion elements,\\
          1000 Flop/element/step,\\
          1 million steps,\\
          CPU speed 10 GFlop/s: total 28,000 CPU hours (3.2 CPU years),
          or 12 days when running on 100 CPUs.
  \end{itemize}
}

\frame[containsverbatim]{ \frametitle{Batch Processing}
  \begin{itemize}
    \item Since simulations take so long we cannot supervise them manually
      \begin{itemize}
        \item Cannot be awake at all times.
        \item A user error can destroy weeks of data.
        \item Supercomputers are expensive; cannot waste valuable resources
              waiting for the next user input.
      \end{itemize}
    \item Need to \emph{plan} simulations carefully ahead of time, then let
          them run automatically.
  \end{itemize}
}

\frame[containsverbatim]{ \frametitle{Batch Processing}
  \begin{itemize}
    \item Need to \emph{plan} simulations carefully ahead of time, then let
          them run automatically...
    \item ...so that each error is only discovered days or weeks later!
    \item Using a supercomputer thus requires much expertise, experience,
          patience and a high tolerance for frustration.
    \item \emph{Using supercomputers are more difficult than it really should
          be}.
  \end{itemize}
}

\section{Supercomputers}
\frame{\frametitle{}\begin{centering}\LARGE\insertsectionhead\\\end{centering}}

\frame[containsverbatim]{ \frametitle{Fast vs.\ Large}
  \begin{itemize}
    \item Supercomputers are not so much fast as they are large.
    \item They are not interactive (like a notebook or workstation), they
          operate in batch mode.
    \item Their hardware is complex.
  \end{itemize}
  \includegraphics[width=4.55cm]{pics/fast}\hspace{0.5em}
  \includegraphics[width=5cm]{pics/large}
}

\frame[containsverbatim]{ \frametitle{Remote Access}
  \begin{itemize}
    \item Supercomputers are located far away, need to use ssh/gsissh to access.
    \item Log in is to \emph{front end (head node)} only, usually a large
          workstation.
    \item Cannot (or should not) use front end to run simulations.
  \end{itemize}
  \includegraphics[width=5.55cm]{pics/hpc-map}\hspace{0.5em}
  \includegraphics[width=4cm]{pics/head-node}
}
\frame[containsverbatim]{ \frametitle{File Systems}
  \begin{itemize}
    \item Supercomputers need large and fast file systems to store simulation
          data often many 100 TByte or PByte.
    \item For management and performance reasons the file systems  are usually
          split into different parts with different properties.
      \begin{description}
        \item[Home directory] GBytes per user, many small files, backed up.
        \item[Data directory] TBytes per user, few large files, backed up,
                              tape backend.
        \item[Scratch directory] No quota, few large files, often automatically
                                 deleted.
      \end{description}
    \item This is done differently on each supercomputer -- read documentation!
  \end{itemize}
}

\frame[containsverbatim]{ \frametitle{Compute Nodes and Interconnect}
  \begin{itemize}
    \item Most supercomputers have a \emph{cluster} architecture with many
          interconnected compute nodes.
    \item Each node may have multiple cores (4 to 64), similar to a
          large workstation typically with 1GByte to 4GByte of memory per core.
    \item Nodes may additionally have GPU's or Intel Xeon Phi's.
    \item Nodes are connected via a low-latency \emph{communication network}
          (e.g. Infiniband or other proprietary technologies).
    \item Overall system has from a few up to 10,000 nodes. The largest
          Supercomputer in the world currently has 3,120,000 cores in 16,000 
	  nodes (Intel Xeon IvyBridge processors and Xeon Phi).
    \item My personal scale:
      \begin{description}
        \item[small] $<2$k cores
        \item[medium] $<20$k cores
        \item[large] $>20$k cores
      \end{description}
  \end{itemize}
}

\frame[containsverbatim]{ \frametitle{Batch System}
  \begin{itemize}
    \item Cannot (or should not) use compute nodes directly.
    \item Need to submit \emph{job} to \emph{batch system}, requesting $N$ 
          nodes for $T$ amount of time.
      \begin{itemize}
        \item ...wait (a few days?)...
        \item ...then the job start to run (most likely while you're asleep)...
        \item ...and then you discover your errors...
      \end{itemize}
    \item There is always a run time limit (e.g.\ 8, 12, 24, 48, 72 hours)
      \begin{itemize}
        \item ...which is inconvenient when you need to run for 2 weeks:
              checkpoint/restart is necessary.
      \end{itemize}
    \item The batch system ensures that jobs get run in an order that
          keeps the supercomputer as busy as possible at all times.
  \end{itemize}
}

\frame[containsverbatim]{ \frametitle{Allocations}
  \begin{itemize}
    \item Need to ensure fair use of supercomputers and prevent individual
          users from monopolising it.
    \item Need to ensure that supercomputer time are used for important
          research and by codes that can run efficiently.
   \item Allocation proposals for the national supercomputers are therefore
         peer reviewed.
   \item The allocation review panel then decides how much time each research
         project will be allowed to use in the coming allocation period.
   \item 1 CPU hour costs about 5 cents (10 cents on Amazon ECC).
   \item With this metric, Super Mike II produces about \$352 worth of CPU time
         every hour. 
  \end{itemize}
  \includegraphics[width=2cm]{xsede}\hspace{0.5em}
  \includegraphics[width=2cm]{loni}\hspace{0.5em}
  \includegraphics[width=2cm]{nersc}
}

\frame[containsverbatim]{ \frametitle{Software}
  \begin{itemize}
    \item Installed/available software is system dependent, not just standard
          Unix systems.
    \item Therefore cannot just install binaries, need to build software for
          each supercomputer.
    \item Installed software typically consists of compilers, MPI libraries,
          Scientific libraries (like GSL, BLAS/LAPACK, PETSC, etc.), perl,
          python, etc.
    \item Most interactions with a supercomputer takes place on the command
          line. Some GUIs, Portals exist, but typically limited in 
          functionality.
  \end{itemize}
}

\frame[containsverbatim]{ \frametitle{Supercomputer Steps}
  \begin{itemize}
    \item Obtain an \emph{Allocation}.
    \item Log in to the \emph{Front End}.
    \item Compile your code on the \emph{Front End}.
    \item Submit your jobs on the \emph{Front End} to the \emph{Batch Queue}.
    \item Simulations then execute on the \emph{Compute Nodes} connected via
          a \emph{Communication Network}.
    \item Data stored in \emph{File Systems} can then be analyzed on the
          supercomputer itself or transferred to other places for analysis.
    \item Data can be stored permanently in \emph{Tape Archives}.
  \end{itemize}
}

\end{document}
