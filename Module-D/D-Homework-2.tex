\documentclass{beamer}

\usepackage[utf8]{inputenc}
%\usepackage{beamerthemesplit}
\usepackage{url}
\usepackage{tikz}
\usepackage{alltt}
\usepackage{listings}
\usepackage{marvosym}
\usepackage{color}
\usepackage[multidot]{grffile}
\usepackage{multirow}
\usepackage{array}
\usepackage{setspace}
\usepackage{hyperref}
\usepackage{verbatim}
\usepackage{fancyvrb}
%\hypersetup{colorlinks=true, linkcolor=blue,  anchorcolor=blue,  
%citecolor=blue, filecolor=blue, menucolor=blue, pagecolor=blue,  
%urlcolor=blue} 
\lstset{keywordstyle=\bfseries\color{brown},
        stringstyle=\ttfamily,
        commentstyle=\color{blue}\textit,
        showstringspaces=false}

\useoutertheme{}
\usetheme{Madrid}
\graphicspath{{pics/}{global/}
{pics/I/}{pics/A1/}{pics/A2/}{pics/A3/}{pics/A4/}{pics/A5/}{pics/A6/}{pics/A7/}
}

\logo{\includegraphics[height=1cm]{ProcessHorizontal}} 

\institute{Center for Computation and Technology\\Louisiana State University, Baton Rouge, LA}

\setbeamertemplate{navigation symbols}{} 

\title{CSC 7700: Scientific Computing}

% We want to use the infolines outer theme because it does not use a lot of
% space, but it also tries to print an institution and the slide
% numbers (which we might not want to show). Therefore, we here redefine the
% footline ourselfes - mostly a copy & paste from
% /usr/share/texmf/tex/latex/beamer/themes/outer/beamerouterthemeinfolines.sty
\defbeamertemplate*{footline}{infolines theme without institution and slide numbers}
{
  \leavevmode%
  \hbox{%
  \begin{beamercolorbox}[wd=.25\paperwidth,ht=2.25ex,dp=1ex,center]{author in head/foot}%
    \usebeamerfont{author in head/foot}\insertshortauthor
  \end{beamercolorbox}%
  \begin{beamercolorbox}[wd=.5\paperwidth,ht=2.25ex,dp=1ex,center]{title in head/foot}%
    \usebeamerfont{title in head/foot}\insertshorttitle
  \end{beamercolorbox}%
  \begin{beamercolorbox}[wd=.25\paperwidth,ht=2.25ex,dp=1ex,center]{date in head/foot}%
    \usebeamerfont{date in head/foot}\insertshortdate{}
  \end{beamercolorbox}}%
  \vskip0pt%
}

% Some useful commands
\newcommand{\abspic}[4]
 {\vspace{ #2\paperheight}\hspace{ #3\paperwidth}\includegraphics[height=#4\paperheight]{#1}\\
  \vspace{-#2\paperheight}\vspace{-#4\paperheight}\vspace{-0.0038\paperheight}}

\newcommand{\picw}[4]{{
 \usebackgroundtemplate{
 \color{black}\vrule width\paperwidth height\paperheight\hspace{-\paperwidth}\hspace{-0.01\paperwidth}
 \hspace{#4\paperwidth}\includegraphics[width=#3\paperwidth, height=\paperheight]{#1}}\logo{}
 \frame[plain]{\frametitle{#2}}
}}
\newcommand{\pic}[2]{\picw{#1}{#2}{}{0}}

\newcommand{\question}[1]{\frame{\frametitle{#1}
 \begin{centering}\Huge #1\\\end{centering}
}}



%\usecolortheme[RGB={200,200,200}]{structure}

\subtitle[Module D]{{\large Module D: Simulations and Application Frameworks}\\*[0.3em]Homework 2: Write a Cactus thorn}
\author[Peter Diener]{Dr Peter Diener}
\date{November 22, 2013}
\usecolortheme[RGB={45,160,140}]{structure}

\begin{document}

\frame{\titlepage}

\section{Homework}
\frame{\frametitle{}\begin{centering}\LARGE\insertsectionhead\\\end{centering}}

\frame[containsverbatim]{ \frametitle{Homework}
  Your homework will be to design and implement a (fairly) simple Cactus
  thorn.
  \begin{itemize}
    \item Physics problem: Find out ``where'' the star from last weeks
          homework assignment is oscillating, i.e.\ what radius contributes
	  most to the kinetic energy of the oscillations.
    \item Method: Calculate the kinetic energy density everywhere in the
          neutron star and figure out where it is largest.
    \item The kinetic energy density is:
          \[
            e_{\,\mathrm{kin}} = \frac{1}{2} \rho v^2 = \frac{1}{2}\rho\,
                                                      (v_x^2+v_y^2+v_z^2),
          \]
          i.e.\ it is large where the density $\rho$ is high and the velocity
          $v$ is high.
  \end{itemize}
}

\frame[containsverbatim]{ \frametitle{Homework}
  \begin{itemize}
    \item To get started use the command: \\
          \texttt{make newthorn} \\
          in the Cactus directory. This will ask you for:
      \begin{description}
        \item[Thorn name\hfill] Pick one you like.
        \item[arrangement\hfill] Choose an existing one or create your own with
                           a name you like.
        \item[Thorn Author Name] Type your name.
        \item[Email Address] Type your email address.
        \item[Add another author? Y/N] Type \texttt{n} or \texttt{N}.
        \item[License\hfill] Type in your favorite source code license (e.g. GPL).
      \end{description}
    \item After this a new template for a thorn will have been created with
	    \texttt{README}, \texttt{configuration.ccl}, 
	    \texttt{interface.ccl}, \texttt{param.ccl} and
          \texttt{schedule.ccl} files and \texttt{doc}, \texttt{par},
          \texttt{src} and \texttt{test} directories to be filled in.
  \end{itemize}
}

\frame[containsverbatim]{ \frametitle{Homework}
  \begin{itemize}
    \item The variables \texttt{rho} (grid function) and \texttt{vel} (3-vector
          gridfunction) are defined in \texttt{EinsteinBase/HydroBase}.
    \item The kinetic energy density is defined at every gridpoint and needs
          to be stored in a grid function.
    \item Write a routine (placed in the \texttt{src} directory) to iterate
          over all grid points\ldots
    \item \ldots and then evaluate the formula from the previous slide
          for each grid point and store the result in the grid function.
    \item It's up to you if you want to use C, C++ or F90. The grid functions
          will be accessed differently depending on your choice.
      \begin{itemize}
        \item F90: look at\\
              {\footnotesize
              \texttt{EinsteinInitialData/GRHydro\_InitData/src/GRHydro\_ShockTube.F90}
              }
        \item C or C++: look at\\
              {\footnotesize
              \texttt{EinsteinInitialData/TOVSolver/src/tov.c}.
              }
      \end{itemize}
  \end{itemize}
}

\frame[containsverbatim]{ \frametitle{Homework}
  \begin{itemize}
    \item \underline{\texttt{interface.ccl}:}\\
      \begin{itemize}
        \item Define the implementation name that the thorn implements.
        \item Inherit from \texttt{HydroBase} where \texttt{rho} and
              \texttt{vel} are defined.
        \item Declare a new grid function for kinetic energy density.
      \end{itemize}
    \item \underline{\texttt{schedule.ccl}:}\\
      \begin{itemize}
        \item Schedule your routine that performs the calculation in the
              \texttt{analysis} schedule bin.
        \item Request storage for the new grid function for this calculation.
      \end{itemize}
    \item \underline{\texttt{param.ccl}:}\\
      \begin{itemize}
        \item This thorn doesn't need any parameters.
      \end{itemize}
    \item \underline{\texttt{configuration.ccl}:}\\
      \begin{itemize}
        \item This thorn doesn't use any special capabilities.
      \end{itemize}
  \end{itemize}
}

\frame[containsverbatim]{ \frametitle{Homework}
  \begin{itemize}
    \item Add your source filename to \texttt{make.code.defn} so that the
          make system will compile your code. Look at other thorns to see how.
    \item Add your new thorn to the thornlist you used to compile the
          Cactus configuration from last week's homework and recompile.
	  Go back and fix problems if the compile fails and try again.
    \item Copy the parameter file from last week's homework to a different name
          and edit it:
      \begin{itemize}
        \item Activate your thorn.
        \item Request 1D output of your new grid function.
      \end{itemize}
    \item Run as you did last week but specify the new parameter file.
    \item Visualise the 1D output. Note that this run uses AMR so the 1D output
          contains data from multiple refinement levels. We are here only
          interested in the output from the finest refinement level (which
          is refinement level 4 in the output file).
  \end{itemize}
}

\frame[containsverbatim]{ \frametitle{Homework}
  \begin{itemize}
    \item Your homework will consist of the a tar ball of your thorn and
          a plot of the kinetic energy density as a function of x. Use
          a gnuplot command like:\\
          \texttt{p '<file\_name>' u 10:((\$3==4 \&\& \$9==1000)?\$13:1/0) w lp}\\
         to extract the relevant data.
    \item Brownie points to whomever can explain how the gnuplot command above
          works.
    \item Homework is due on December 6, 2013.
  \end{itemize}
}

\end{document}
