\documentclass{beamer}

\usepackage[utf8]{inputenc}
%\usepackage{beamerthemesplit}
\usepackage{url}
\usepackage{tikz}
\usepackage{alltt}
\usepackage{listings}
\usepackage{marvosym}
\usepackage{color}
\usepackage[multidot]{grffile}
\usepackage{multirow}
\usepackage{array}
\usepackage{setspace}
\usepackage{hyperref}
\usepackage{verbatim}
\usepackage{fancyvrb}
%\hypersetup{colorlinks=true, linkcolor=blue,  anchorcolor=blue,  
%citecolor=blue, filecolor=blue, menucolor=blue, pagecolor=blue,  
%urlcolor=blue} 
\lstset{keywordstyle=\bfseries\color{brown},
        stringstyle=\ttfamily,
        commentstyle=\color{blue}\textit,
        showstringspaces=false}

\useoutertheme{}
\usetheme{Madrid}
\graphicspath{{pics/}{global/}
{pics/I/}{pics/A1/}{pics/A2/}{pics/A3/}{pics/A4/}{pics/A5/}{pics/A6/}{pics/A7/}
}

\logo{\includegraphics[height=1cm]{ProcessHorizontal}} 

\institute{Center for Computation and Technology\\Louisiana State University, Baton Rouge, LA}

\setbeamertemplate{navigation symbols}{} 

\title{CSC 7700: Scientific Computing}

% We want to use the infolines outer theme because it does not use a lot of
% space, but it also tries to print an institution and the slide
% numbers (which we might not want to show). Therefore, we here redefine the
% footline ourselfes - mostly a copy & paste from
% /usr/share/texmf/tex/latex/beamer/themes/outer/beamerouterthemeinfolines.sty
\defbeamertemplate*{footline}{infolines theme without institution and slide numbers}
{
  \leavevmode%
  \hbox{%
  \begin{beamercolorbox}[wd=.25\paperwidth,ht=2.25ex,dp=1ex,center]{author in head/foot}%
    \usebeamerfont{author in head/foot}\insertshortauthor
  \end{beamercolorbox}%
  \begin{beamercolorbox}[wd=.5\paperwidth,ht=2.25ex,dp=1ex,center]{title in head/foot}%
    \usebeamerfont{title in head/foot}\insertshorttitle
  \end{beamercolorbox}%
  \begin{beamercolorbox}[wd=.25\paperwidth,ht=2.25ex,dp=1ex,center]{date in head/foot}%
    \usebeamerfont{date in head/foot}\insertshortdate{}
  \end{beamercolorbox}}%
  \vskip0pt%
}

% Some useful commands
\newcommand{\abspic}[4]
 {\vspace{ #2\paperheight}\hspace{ #3\paperwidth}\includegraphics[height=#4\paperheight]{#1}\\
  \vspace{-#2\paperheight}\vspace{-#4\paperheight}\vspace{-0.0038\paperheight}}

\newcommand{\picw}[4]{{
 \usebackgroundtemplate{
 \color{black}\vrule width\paperwidth height\paperheight\hspace{-\paperwidth}\hspace{-0.01\paperwidth}
 \hspace{#4\paperwidth}\includegraphics[width=#3\paperwidth, height=\paperheight]{#1}}\logo{}
 \frame[plain]{\frametitle{#2}}
}}
\newcommand{\pic}[2]{\picw{#1}{#2}{}{0}}

\newcommand{\question}[1]{\frame{\frametitle{#1}
 \begin{centering}\Huge #1\\\end{centering}
}}



%\usecolortheme[RGB={200,200,200}]{structure}

\subtitle[Module D]{{\large Module D: Simulations and Application Frameworks}\\*[0.3em]Lecture 3: Cactus Framework Architecture}
\author[Peter Diener]{Dr Peter Diener}
\date{November 22, 2013}
\usecolortheme[RGB={132,186,75}]{structure}
\definecolor{cactusgreen}{RGB}{132,186,75}

\begin{document}

\frame{\titlepage}

\section*{Outline}
\frame{\tableofcontents}

\section{Goals}
\frame{\frametitle{}\begin{centering}\LARGE\insertsectionhead\\\end{centering}}

\frame[containsverbatim]{ \frametitle{Goals}
  \begin{itemize}
    \item Lecture 1 introduced:
      \begin{itemize}
        \item The concept of a simulation and it's ingredients.
        \item Super computers from the application scientist's point of view.
      \end{itemize}
    \item Lecture 2 introduced:
      \begin{itemize}
        \item Parallelisation: data structures, load balancing, domain
              decomposition.
        \item Software Engineering: multi-physics simulations, large projects,
              distributed code development.
      \end{itemize}
    \item In this lecture we will discuss:
      \begin{itemize}
        \item The component model as software architecture for real-world
              simulation codes.
        \item The Cactus Software Framework as a specific example.
      \end{itemize}
  \end{itemize}
}

\section{Summary}
\frame{\frametitle{}\begin{centering}\LARGE\insertsectionhead\\\end{centering}}

\frame[containsverbatim]{ \frametitle{Summary}
  \begin{itemize}
    \item To go from physics to a simulation, one usually
      \begin{enumerate}
        \item Finds a mathematical model (e.g.\ PDEs) expressing the physics.
        \item Discretise the model (e.g.\ PDEs).
        \item Implement the discretised equations on a supercomputer
      \end{enumerate}
    \item Many simulation codes have a similar structure.
    \item Many supercomputers have a similar architecture.
  \end{itemize}
}

\frame[containsverbatim]{ \frametitle{Summary}
  \begin{itemize}
    \item Parallel algorithm are necessary due to size of the problems (memory)
          and computational cost (CPU time).
    \item MPI is the tool of choice. Requires domain decomposition, advanced
          data structures and load balancing algorithms.
    \item A component model is necessary to develop complicated multi-physics
          codes using geographically distributed code developers.
    \item A \emph{framework} provides the glue between components.
    \item We introduced the Einstein Toolkit as a real world example.
  \end{itemize}
}

\section{Cactus Software Framework}
\frame{\frametitle{}\begin{centering}\LARGE\insertsectionhead\\\end{centering}}

\frame[containsverbatim]{ \frametitle{Cactus}
  \begin{minipage}[t]{8cm}
    \vspace{0pt}
    \begin{itemize}
      \item Open source HPC software framework developed at LSU/AEI/Caltech/PI.
      \item First version designed at AEI in 1997 in the field of Numerical
            Relativity to address simulation code difficulties described
            earlier.
      \item Redesigned in 1999 to allow it to be more easily used by other
            fields of science.
      \item Highly portable (from notebook to supercomputers).
      \item Highly parallel (tested on $>130,000$ cores).
    \end{itemize}
  \end{minipage}
  \hfill
  \begin{minipage}[t]{3.0cm}
    \vspace{0pt}
    \includegraphics[width=2.7cm]{pics/bincactus2}
  \end{minipage}
}

\frame[containsverbatim]{ \frametitle{Cactus Design}
  \begin{itemize}
    \item The core of the framework (\emph{the flesh}) is lean. It only provides
          \emph{glue} between components. It performs no ``real work''.
    \item The components (\emph{the thorns}) perform all the work, both
          computational and physics.
    \item It contains many standard thorns such as providing coordinate
          systems, standard boundary conditions, MPI drivers, efficient
          parallel I/O, etc.
  \end{itemize}
}

\frame { \frametitle{Covers}
 \hspace{0cm}\includegraphics[height=3.5cm]{covers/ACMCommunications}
 \hspace{1cm}\includegraphics[height=3.5cm]{covers/FutureTechPhysCosm}
 \hspace{1cm}\includegraphics[height=3.5cm]{covers/Gravity}\\\vspace{-1.5cm}
 \hspace{1cm}\includegraphics[height=3.5cm]{covers/max-planck-research2}
 \hspace{1cm}\includegraphics[height=3.5cm]{covers/nsffy2006budgetcover}
 \hspace{1cm}\includegraphics[height=3.5cm]{covers/mmcover}\\\vspace{-1.5cm}
 \hspace{2cm}\includegraphics[height=3.5cm]{covers/visprocessingtensor}
 \hspace{1.5cm}\includegraphics[height=3.5cm]{covers/PetascaleComputingBook}
 \hspace{1cm}\includegraphics[height=3.5cm]{covers/NR689}
}

\frame { \frametitle{Many arrangements with many modules...}
  \begin{tabular}{ll}
    CactusBase& Basic utility and interface thorns\\
    CactusBench& Benchmark utility thorns\\
    CactusConnect& Network utility thorns\\
    CactusElliptic& Elliptic solvers / interface thorns\\
    CactusExamples& Example thorns\\
    CactusExternal& External library interface thorns\\
    CactusIO& General I/O thorns\\
    CactusNumerical& General numerical methods\\
    CactusPUGH& Cactus Unigrid Driver thorn\\
    CactusPUGHIO& I/O thorns specific for PUGH driver\\
    CactusTest& Thorns for Cactus testing\\
    CactusUtils& Misc. utility thorns\\
    CactusWave& Wave example thorns\\
  \end{tabular}
}

\frame { \frametitle{Application View}
The structure of an application that is built upon the Cactus computational
framework
\begin{figure}
\vspace{0.0cm}
\hspace{0.0cm}
\centering
\includegraphics[height=0.5\textwidth]{cactus_appview}
\end{figure}
}

\frame { \frametitle{Other Frameworks}
  \begin{itemize}
    \item Several other HPC simulation frameworks or framework-like
          architectures exist.
      \begin{itemize}
        \item However, most are libraries rather than frameworks.
      \end{itemize}
    \item Frameworks are also common in other software areas.
      \begin{itemize}
        \item e.g.\ KDE (linux desktop) and Eclipse (software development
           environment).
      \end{itemize}
  \end{itemize}
}

\frame { \frametitle{Code assembly}
  \begin{itemize}
    \item Different from a ``traditional'' program. In a framework only the
          end user controls what components (plugins) are present (active).
    \item In Cactus, this is handled via \emph{thorn lists} defining the
          names and download locations of all thorns to be compiled into
          the executable.
    \item At run time, thorns can be activated if needed for a given simulation.
  \end{itemize}
}

\frame[containsverbatim] { \frametitle{Thornlists}
 \begin{itemize}
  \item List of thorn names
  \item Corresponding download methods and locations (optional)
  \item Supported download methods:
   \begin{itemize}
    \item CVS / Subversion / Git / Mercurial
    \item http / https / ftp
%    \item Ignore
   \end{itemize}
   \item Example:
 \end{itemize}
 \centering\begin{minipage}{0.5\linewidth}\tiny
\begin{verbatim}
 !CRL_VERSION = 1.0

 # Cactus Flesh
 !TARGET   = $ROOT
 !TYPE     = svn
 !URL      = http://svn.cactuscode.org/flesh/trunk
 !CHECKOUT = Cactus
 !NAME     = .

 # Cactus thorns
 !TARGET   = Cactus/arrangements
 !TYPE     = svn
 !URL      = http://svn.cactuscode.org/arrangements/$1/$2/trunk
 !CHECKOUT = 
 CactusBase/Boundary
 CactusBase/CartGrid3D
 CactusBase/CoordBase
 CactusBase/IOASCII
 CactusBase/IOBasic
\end{verbatim}
\end{minipage}\\
}

\frame { \frametitle{Component structure}
  \begin{itemize}
    \item A component (thorn) has:
      \begin{itemize}
        \item An \emph{interface} that contains its external specification,
              describing how it looks to the framework and to other thorns.
        \item An \emph{implementation} that defines how it actually works,
              including documentation and test cases.
      \end{itemize}
  \end{itemize}
}

\frame { \frametitle{Thorn Structure}
Inside view of a plug-in module, or thorn for Cactus
\begin{figure}
\vspace{0.0cm}
\hspace{0.0cm}
\centering
\includegraphics[height=0.55\textwidth]{thorn_arch}
\end{figure}
}

\frame[containsverbatim] { \frametitle{Thorn Structure}
 Directory structure:\\
 \centering\begin{minipage}{0.6\linewidth}
 \begin{alltt}
\textcolor{gray}{Cactus
`-- arrangements
    `-- Introduction}
        `-- HelloWorld
            \textcolor{blue}{|-- interface.ccl
            |-- param.ccl
            |-- schedule.ccl}
            |-- README
            |-- doc
            |   `-- documentation.tex
            |-- src
            |   |-- HelloWorld.c
            |   `-- make.code.defn
            |-- test
            `-- utils
 \end{alltt}
 \end{minipage}\\
}

\frame { \frametitle{Thorn Specification}
  Four configuration files per thorn:
  \begin{itemize}
    \item<1-> \textcolor{blue}{interface.ccl} declares:
      \begin{itemize}
        \item an `implementation' name
        \item inheritance relationships between thorns
        \item Thorn variables
        \item Global functions, both provided and used
      \end{itemize}
    \item<2-> \textcolor{blue}{schedule.ccl} declares:
      \begin{itemize}
        \item When the flesh should schedule which functions
        \item When which variables should be allocated/freed
        \item Which variables should be syncronized when
      \end{itemize}
    \item<3-> \textcolor{blue}{param.ccl} declares:
      \begin{itemize}
        \item Runtime parameters for the thorn
        \item Use/extension of parameters of other thorns
      \end{itemize}
    \item<4-> \textcolor{blue}{configuration.ccl (optional)} declares:
      \begin{itemize}
        \item Capabilities provided by or used by the thorn
        \item Capabilities are usually related to external libraries
      \end{itemize}
  \end{itemize}
}

\frame { \frametitle{Grid Functions}
  \begin{itemize}
    \item A discretised physical quantity (e.g.\ density, pressure, 
          velocity\ldots) lives on a grid function.
    \item If a regular discretisation is used these are essentially distrubuted
          arrays (hence ``grid'')
      \begin{itemize}
        \item Could also be a graph, tree, etc.\ instead.
      \end{itemize}
    \item Grid functions are fundamental data types in a simulation.
  \end{itemize}
}

\frame { \frametitle{Example of Cactus Grid Function Shapes}
  \begin{minipage}[t]{5cm}
    \vspace{0pt}
    \begin{center}
      Adaptive Mesh Refinement and Multi-Block\\
      \includegraphics[width=4.8cm]{amr-multipatch}
    \end{center}
  \end{minipage}\hfill
  \begin{minipage}[t]{6cm}
    \vspace{0pt}
    \begin{center}
      Multi-Block with distorted coordinate systems\\
      \includegraphics[width=5.8cm]{distorted-multipatch}
    \end{center}
  \end{minipage}
}

\frame[containsverbatim]{ \frametitle{\texttt{interface.ccl}}
  Variables:
  \begin{itemize}
    \item Flesh needs to know about thorn variables for which it has to care about
          allocation, parallelism, inter-thorn use
    \item Scopes: public, private
    \item Many different basic types (double, integer, string, \ldots
    \item Different `group types' (grid functions, grid arrays, scalars,
        \ldots)
    \item Different `tags' (not to be checkpointed, vector types, \ldots)
  \end{itemize}
}

\frame[containsverbatim] { \frametitle{Syntax of \texttt{interface.ccl}}
\begin{verbatim}
IMPLEMENTS: <interface name>
INHERITS: <interface name> ...

[PUBLIC:|PRIVATE:]

[REAL|COMPLEX|INT] <group name> TYPE=[gf|array]
        TIMELEVELS=<number> [DIM=... SIZE=...]
{
  <variable name>
  ...
} <description>
\end{verbatim}
}

\frame[containsverbatim] { \frametitle{Syntax of \texttt{interface.ccl} cont.}
\begin{verbatim}
[REAL|COMPLEX|INT|POINTER] FUNCTION <function name> (
       [REAL|COMPLEX|INT|STRING|POINTER] \
               [ARRAY] [IN|OUT] <argument name>,
       ...
       )

[USES|REQUIRES] FUNCTION <function name>

PROVIDES FUNCTION <function name>
        WITH <implementation name> LANG [C|FORTRAN]
\end{verbatim}
}

\frame[containsverbatim] { \frametitle{Example \texttt{interface.ccl}}
\begin{verbatim}
IMPLEMENTS: wavetoy

INHERITS: grid

PUBLIC:

REAL scalarevolve TYPE=gf TIMELEVELS=3
{
  phi
} "The evolved scalar field"
\end{verbatim}
}

\frame[containsverbatim] { \frametitle{Example \texttt{interface.ccl} cont.}
\begin{verbatim}
CCTK_INT FUNCTION Boundary_SelectVarForBC (
        CCTK_POINTER_TO_CONST IN cctkGH,
        CCTK_INT IN faces,
        CCTK_INT IN boundary_width,
        CCTK_INT IN options_handle,
        CCTK_STRING IN var_name,
        CCTK_STRING IN bc_name
        )

REQUIRES FUNCTION Boundary_SelectVarForBC
\end{verbatim}
}

\frame { \frametitle{Execution Control Scheduling}
  \begin{itemize}
    \item Components are developed independently yet need to execute in a
          certain order.
      \begin{itemize}
        \item e.g.\ calculate pressure first then forces from pressure gradient.
      \end{itemize}
    \item Don't want end user to have to specify this. This would require
          ``manual assembly'' of the components. The end user probably doesn't
          understand the thorn details.
  \end{itemize}
}

\frame[containsverbatim]{ \frametitle{\texttt{schedule.ccl}}
  \begin{itemize}
    \item Flesh contains a flexible rule based scheduler
    \item Order is prescribed in \verb|schedule.ccl|
    \item Scheduler also handles when variables are allocated, freed or
          synchronized between parallel processes
    \item Functions or groups of functions can be
     \begin{itemize}
      \item grouped and whole group scheduled
      \item scheduled before or after each other
      \item scheduled depending on parameters
      \item scheduled while some condition is true
     \end{itemize}
    \item Flesh scheduler sorts all rules and flags an error for inconsistent
          schedule requests
    \item Note: The scheduler is in the process of being revamped
  \end{itemize}
}

\frame[containsverbatim]{ \frametitle{Schedule Groups}
  \begin{description}
     \item[CCTK\_STARTUP] For routines, run before the grid hierarchy
                          is set up, for example function registration.
     \item[CCTK\_PARAMCHECK] For routines that check parameter combinations,
                             routines registered here only have access to the
                             grid size and the parameters.
     \item[CCTK\_BASEGRID] Responsible for setting up coordinates, etc.
     \item[CCTK\_INITIAL] For generating initial data.
     \item[CCTK\_POSTINITIAL] Tasks which must be applied after initial data is
                              created.
     \item[CCTK\_PRESTEP] Stuff done before the evolution step.
     \item[CCTK\_EVOL] The evolution step.
     \item[CCTK\_POSTSTEP] Stuff done after the evolution step.
     \item[CCTK\_ANALYSIS] For analysing data.
  \end{description}
}

\frame { \frametitle{\texttt{schedule.ccl} cont.}
  \begin{itemize}
  \item Hierarchy: using schedule groups leads to a \emph{schedule
      tree}
  \item Execution order: can schedule \texttt{BEFORE} or
    \texttt{AFTER} other items\\
    e.g.: take time step after calculating RHS
  \item Loops: can schedule \texttt{WHILE} a condition is true\\
    e.g.: loop while error is too large
  \item Conditions: can schedule \texttt{if} a parameter is set\\
    e.g.: choose between boundary conditions
  \item Perform analysis at run time: \texttt{TRIGGERS} statements:
    call routine only if result is needed for I/O
  \end{itemize}
}

\frame { \frametitle{Example scheduling tree}
 \centering\includegraphics[height=7cm]{pics/schedule_tree}\hspace{2cm}
           \includegraphics[height=7cm]{pics/schedule_tree_example}\\
}

\frame[containsverbatim] { \frametitle{Syntax of \texttt{schedule.ccl}}
\begin{verbatim}
SCHEDULE <function name> [AT <schedule bin>|
                          IN <schedule group>]
{
  LANG: [C|Fortran]
  SYNC: <group name> ...
} <description>

SCHEDULE GROUP <name> [AT <schedule bin>|
                       IN <schedule group>]
{
} <description>}

STORAGE: <group name>[timelevels] ...
\end{verbatim}
}

\frame[containsverbatim] { \frametitle{Example \texttt{schedule.ccl}}
\begin{verbatim}
SCHEDULE WaveToyC_Evolution AT evol
{
  LANG: C
} "Evolution of 3D wave equation"

SCHEDULE GROUP WaveToy_Boundaries AT evol \
               AFTER WaveToyC_Evolution
{
} "Boundaries of 3D wave equation"

STORAGE: scalarevolve[3]
\end{verbatim}
}

\frame { \frametitle{Simulation Specification}
  \begin{itemize}
    \item Most thorns have some parameters that are only set at run time.
      \begin{itemize}
        \item e.g.\ number of grid points, initial data model, boundary
              conditions, number of time steps, output frequency,\ldots
      \end{itemize}
    \item End user has to specify these parameters to describe the complete
          simulation setup when choosing which thorns to activate.
  \end{itemize}
}

\frame[containsverbatim]{ \frametitle{\texttt{param.ccl}}
  \begin{itemize}
    \item Definition of parameters
    \item Scopes: Global, Restricted, Private
    \item Thorns can use and extend each others parameters
    \item Different types (double, integer, string, keyword, \ldots)
    \item Range checking and validation
    \item Steerablility at runtime
  \end{itemize}
}

\frame[containsverbatim] { \frametitle{Syntax of \texttt{param.ccl}}
\begin{verbatim}
[SHARES: <implementation>]

[PUBLIC:|RESTRICTED:|PRIVATE:]

[BOOLEAN|KEYWORD|INT|REAL|STRING]
    <parameter name> <description> [STEERABLE=...]
{
    <allowed value>             :: <description>
    <lower bound>:<upper bound> :: <description>
    <pattern>                   :: <description>
    ...
} <default value>
\end{verbatim}
}

\frame[containsverbatim] { \frametitle{Example \texttt{param.ccl}}
\begin{verbatim}
SHARES: grid
USES KEYWORD type

PRIVATE:
KEYWORD initial_data "Type of initial data"
{
    "plane"    :: "Plane wave"
    "gaussian" :: "Gaussian wave"
} "gaussian"

REAL radius "The radius of the gaussian wave"
{
    0:* :: "Positive"
} 0.0
\end{verbatim}
}

\frame { \frametitle{Hello World Thorn}
  \begin{itemize}
    \item Begin with a simple C program.
    \item Design the thorn CCL files.
    \item Convert the source code to Cactus source.
    \item Write a parameter file (it needs to activate the thorn).
    \item Run it and look at the output.
  \end{itemize}
}

\frame[containsverbatim]{ \frametitle{Hello World, Standalone}
 Standalone in C:
 \begin{verbatim}
  #include <stdio.h>
  int main(void)
  {
    printf("Hello World!\n");
    return 0;
  }
 \end{verbatim}
}

\frame[containsverbatim]{ \frametitle{Hello World Thorn}
 \begin{itemize}
  \item \verb|interface.ccl|:
   \begin{verbatim}
   implements: HelloWorld\end{verbatim}
 \item \verb|schedule.ccl|:
  \begin{verbatim}
   schedule HelloWorld at CCTK_EVOL
   {
     LANG: C
   } "Print Hello World message"\end{verbatim}
  \item \verb|param.ccl|: empty
  \item \verb|README|: \tiny
   \begin{verbatim}
      Cactus Code Thorn HelloWorld
      Author(s)    : Frank Löffler <knarf@cct.lsu.edu>
      Maintainer(s): Frank Löffler <knarf@cct.lsu.edu>
      Licence      : GPL
      --------------------------------------------------------------------------

      1. Purpose

      Example thorn for tutorial Introduction to Cactus
\end{verbatim}
 \end{itemize}
}

\frame[containsverbatim]{ \frametitle{Hello World Thorn cont.}
 \begin{itemize}
  \item \verb|src/HelloWorld.c|:
   \begin{verbatim}
    #include "cctk.h"
    #include "cctk_Arguments.h"

    void HelloWorld(CCTK_ARGUMENTS)
    {
      DECLARE_CCTK_ARGUMENTS
      CCTK_INFO("Hello World!");
      return;
    }\end{verbatim}
  \item \verb|src/make.code.defn|:
    \begin{verbatim}
     SRCS = HelloWorld.c\end{verbatim}
 \end{itemize}
}

\frame[containsverbatim]{ \frametitle{Hello World Thorn}
 \begin{itemize}
  \item parameter file:
   \begin{verbatim}
    ActiveThorns = "HelloWorld"
    Cactus::cctk_itlast = 10\end{verbatim}
  \item run: \verb|[mpirun] <cactus executable> <parameter file>|
 \end{itemize}
}

\frame[containsverbatim]{ \frametitle{Hello World Thorn}
 \begin{itemize}
  \item Screen output:\tiny
   \begin{verbatim}       10                                  
  1   0101       ************************  
  01  1010 10      The Cactus Code V4.0    
 1010 1101 011      www.cactuscode.org     
  1001 100101    ************************  
    00010101                               
     100011     (c) Copyright The Authors  
      0100      GNU Licensed. No Warranty  
      0101                                 

Cactus version:    4.0.b17
Compile date:      Sep 08 2011 (13:15:01)
Run date:          Sep 08 2011 (13:15:54-0500)
[...]

Activating thorn Cactus...Success -> active implementation Cactus
Activation requested for 
--->HelloWorld<---
Activating thorn HelloWorld...Success -> active implementation HelloWorld
--------------------------------------------------------------------------------
INFO (HelloWorld): Hello World!
INFO (HelloWorld): Hello World!
INFO (HelloWorld): Hello World!
INFO (HelloWorld): Hello World!
[...] 6x
--------------------------------------------------------------------------------
Done.
   \end{verbatim}
 \end{itemize}
}

\frame[containsverbatim]{ \frametitle{Cactus Summary}
 \begin{itemize}
    \item We introduced the Cactus framework.
    \item Thorns have implementation (regular code) and interface (ccl) files.
    \item Introduced ccl file syntax. See users' guide, reference manual and
          examples for details.
      \begin{itemize}
        \item \texttt{make UsersGuide}
        \item \texttt{make ReferenceManual}
	\item Resulting pdf files will be in \texttt{Cactus/doc}
      \end{itemize}
    \item A parameter file is needed to run a simulation. The parameter
          file also activates the participating thorns.
 \end{itemize}
}

\end{document}
