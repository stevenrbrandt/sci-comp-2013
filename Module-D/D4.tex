\documentclass{beamer}

\usepackage[utf8]{inputenc}
%\usepackage{beamerthemesplit}
\usepackage{url}
\usepackage{tikz}
\usepackage{alltt}
\usepackage{listings}
\usepackage{marvosym}
\usepackage{color}
\usepackage[multidot]{grffile}
\usepackage{multirow}
\usepackage{array}
\usepackage{setspace}
\usepackage{hyperref}
\usepackage{verbatim}
\usepackage{fancyvrb}
%\hypersetup{colorlinks=true, linkcolor=blue,  anchorcolor=blue,  
%citecolor=blue, filecolor=blue, menucolor=blue, pagecolor=blue,  
%urlcolor=blue} 
\lstset{keywordstyle=\bfseries\color{brown},
        stringstyle=\ttfamily,
        commentstyle=\color{blue}\textit,
        showstringspaces=false}

\useoutertheme{}
\usetheme{Madrid}
\graphicspath{{pics/}{global/}
{pics/I/}{pics/A1/}{pics/A2/}{pics/A3/}{pics/A4/}{pics/A5/}{pics/A6/}{pics/A7/}
}

\logo{\includegraphics[height=1cm]{ProcessHorizontal}} 

\institute{Center for Computation and Technology\\Louisiana State University, Baton Rouge, LA}

\setbeamertemplate{navigation symbols}{} 

\title{CSC 7700: Scientific Computing}

% We want to use the infolines outer theme because it does not use a lot of
% space, but it also tries to print an institution and the slide
% numbers (which we might not want to show). Therefore, we here redefine the
% footline ourselfes - mostly a copy & paste from
% /usr/share/texmf/tex/latex/beamer/themes/outer/beamerouterthemeinfolines.sty
\defbeamertemplate*{footline}{infolines theme without institution and slide numbers}
{
  \leavevmode%
  \hbox{%
  \begin{beamercolorbox}[wd=.25\paperwidth,ht=2.25ex,dp=1ex,center]{author in head/foot}%
    \usebeamerfont{author in head/foot}\insertshortauthor
  \end{beamercolorbox}%
  \begin{beamercolorbox}[wd=.5\paperwidth,ht=2.25ex,dp=1ex,center]{title in head/foot}%
    \usebeamerfont{title in head/foot}\insertshorttitle
  \end{beamercolorbox}%
  \begin{beamercolorbox}[wd=.25\paperwidth,ht=2.25ex,dp=1ex,center]{date in head/foot}%
    \usebeamerfont{date in head/foot}\insertshortdate{}
  \end{beamercolorbox}}%
  \vskip0pt%
}

% Some useful commands
\newcommand{\abspic}[4]
 {\vspace{ #2\paperheight}\hspace{ #3\paperwidth}\includegraphics[height=#4\paperheight]{#1}\\
  \vspace{-#2\paperheight}\vspace{-#4\paperheight}\vspace{-0.0038\paperheight}}

\newcommand{\picw}[4]{{
 \usebackgroundtemplate{
 \color{black}\vrule width\paperwidth height\paperheight\hspace{-\paperwidth}\hspace{-0.01\paperwidth}
 \hspace{#4\paperwidth}\includegraphics[width=#3\paperwidth, height=\paperheight]{#1}}\logo{}
 \frame[plain]{\frametitle{#2}}
}}
\newcommand{\pic}[2]{\picw{#1}{#2}{}{0}}

\newcommand{\question}[1]{\frame{\frametitle{#1}
 \begin{centering}\Huge #1\\\end{centering}
}}



%\usecolortheme[RGB={200,200,200}]{structure}

\subtitle[Module D]{{\large Module D: Simulations and Application Frameworks}\\*[0.3em]Lecture 4: Getting Science Out of Computing}
\author[Peter Diener]{Dr Peter Diener}
\date{November 22, 2013}
\usecolortheme[RGB={75,132,186}]{structure}
\definecolor{cactusgreen}{RGB}{132,186,75}

\begin{document}

\frame{\titlepage}

\section*{Outline}
\frame{\tableofcontents}

\section{Goals}
\frame{\frametitle{}\begin{centering}\LARGE\insertsectionhead\\\end{centering}}

\frame[containsverbatim]{ \frametitle{Goals}
  \begin{itemize}
    \item Lecture 1 introduced:
      \begin{itemize}
        \item The concept of a simulation and it's ingredients.
        \item Supercomputers from the application scientist's point of view.
      \end{itemize}
    \item Lecture 2 introduced:
      \begin{itemize}
        \item Parallelization: data structures, load balancing, domain
              decomposition.
        \item Software Engineering: multi-physics simulations, large projects,
              distributed code development.
      \end{itemize}
    \item Lecture 3 introduced:
      \begin{itemize}
        \item The component model as software architecture for real-world
              simulation codes.
        \item The Cactus Software Framework as a specific example.
      \end{itemize}
    \item In this lecture we will discuss:
      \begin{itemize}
        \item Additional framework concepts.
        \item Scientific programming.
      \end{itemize}
  \end{itemize}
}

\section{Summary}
\frame{\frametitle{}\begin{centering}\LARGE\insertsectionhead\\\end{centering}}

\frame[containsverbatim]{ \frametitle{Summary}
  \begin{itemize}
    \item To go from physics to a simulation, one usually
      \begin{enumerate}
        \item Finds a mathematical model (e.g.\ PDEs) expressing the physics.
        \item Discretises the model (e.g.\ PDEs).
        \item Implements the discretized equations on a supercomputer
      \end{enumerate}
    \item Many simulation codes have a similar structure.
    \item Many supercomputers have a similar architecture.
  \end{itemize}
}

\frame[containsverbatim]{ \frametitle{Summary}
  \begin{itemize}
    \item Parallel algorithms are necessary due to size of the problems (memory)
          and computational cost (CPU time).
    \item MPI is the tool of choice. Requires domain decomposition, advanced
          data structures and load balancing algorithms.
    \item A component model is necessary to develop complicated multi-physics
          codes using geographically distributed code developers.
    \item A \emph{framework} provides the glue between components.
    \item We introduced the Einstein Toolkit as a real world example.
  \end{itemize}
}

\frame[containsverbatim]{ \frametitle{Summary}
  \begin{itemize}
    \item We introduced the Cactus framework.
    \item Applications consist of many components (\emph{thorns}) glued 
          together by the framework (\emph{flesh}).
    \item Cactus provides the main program while components are libraries.
    \item The end user can mix and match the thorns necessary for a specific
          problem and control which thorns are active at runtime.
    \item Thorns have implementation (regular code) and interface (ccl) files.
    \item Thorns ``talk'' to each other only through well-defined interfaces
          and an API provided by the flesh.
    \item The MPI parallellisation issues are (mostly) hidden from the
          application programmer (\texttt{SYNC} statements in schedule
          determines ghost zone updates).
  \end{itemize}
}

\section{Additional Framework Concepts}
\frame{\frametitle{}\begin{centering}\LARGE\insertsectionhead\\\end{centering}}

\frame[containsverbatim]{ \frametitle{Cactus: Driver Thorn}
  \begin{itemize}
    \item A \emph{driver} is a special thorn in Cactus that implements
          parallelism and memory management.
    \item The driver implements the ``grid function'' data type (as well as
          ``grid arrays'').
    \item This \emph{externalizes parallelism} so that other thorns don't have
          to implement parallel algorithms
    \item However, this places certain restrictions onto other thorns.
    \item There must be exactly one driver active (standard Cactus driver
          is PUGH).
    \item The driver can provide advanced discretisation methods, such as AMR
          or multi-block (e.g.\ the \emph{Carpet} driver).
    \item The driver can be based on an existing parallel library (e.g.\ Chombo
          or Samrai).
    \item The driver (or closely related thorn) also provides I/O.
  \end{itemize}
}

\section{Application efficiency}
\frame{\frametitle{}\begin{centering}\LARGE\insertsectionhead\\\end{centering}}

\frame[containsverbatim]{ \frametitle{Data Access}
  \begin{itemize}
    \item Simulations handle large data sets (previous example: 1 billion
          elements).
    \item Cannot easily copy data:
      \begin{itemize}
        \item Not enough memory.
        \item It takes too much time.
      \end{itemize}
    \item If possible, each process must compute with the data it owns
          (``bring computation to data'').
    \item In Cactus, work routines are called on each process with access to
          the data owned by the process.
  \end{itemize}
}

\frame[containsverbatim]{ \frametitle{Data Sharing}
  \begin{itemize}
    \item Different components may need to access the same data.
      \begin{itemize}
        \item Example: A spacetime evolution thorn needs access to the
              stress energy tensor and a hydrodynamics evolution thorn
              needs access to the spacetime metric.
      \end{itemize}
    \item If components are very independent, data need to be copied.
    \item If data cannot be copied, the components must interact in
          some (non-trivial) way.
    \item In Cactus this is done by inheritance: A thorn can have direct
          access to another thorn's data.
  \end{itemize}
}

\frame[containsverbatim]{ \frametitle{Component Coupling}
  \begin{itemize}
    \item How closely are components coupled in a framework?
      \begin{description}
        \item[No Coupling:] Independently executing programs. Data 
                           ``sharing'' requires writing/copying/reading files.
        \item[Loose Coupling:] Independent data management and parallelism
                               in each component. Data sharing requires
                               memory transfers.
        \item[Tight Coupling:] Data are managed outside of components (or by
                               a special component). Data sharing is efficient
                               (components share access to the same memory), 
                               but components need to rely on an external
                               data manager.
      \end{description}
  \end{itemize}
}

\frame[containsverbatim]{ \frametitle{Component Coupling}
  \begin{center}
    \includegraphics[width=11cm]{component_interoperability}
  \end{center}
}

\frame[containsverbatim]{ \frametitle{Component Safety}
  \begin{itemize}
    \item Efficient data sharing between components requires running in the
          same address space.
    \item This means that components can (accidentally?) modify each other's
          data. E.g.\ errors (such as array index out of bounds) can propagate
          between components.
    \item Compile time access control and coding standards can provide some
          safety.
  \end{itemize}
}

\frame[containsverbatim]{ \frametitle{Additional Framework Concepts Summary}
  \begin{itemize}
    \item Many simulation frameworks with many different designs exist.
    \item Fundamental design question is: How tight are components coupled?
    \item Tight coupling requires shared data management between components.
    \item Trade-off between independence/ease-of-programming/safety and
          efficiency.
  \end{itemize}
}

\section{Scientific Programming}
\frame{\frametitle{}\begin{centering}\LARGE\insertsectionhead\\\end{centering}}

\frame[containsverbatim]{ \frametitle{Shared Code Development}
  \begin{itemize}
    \item Developing a large code as a group (or community) is different
          from small-scale programming.
      \begin{itemize}
        \item There is old code ($>10$ years old) that ``belongs to nobody''.
        \item People use ``your'' code without understanding it.
        \item People make changes to ``your'' code without understanding it.
      \end{itemize}
    \item Best not to have ``your'' or ``my'' code. Instead share 
          responsibility.
    \item Program defensively, so that wrong usage is (always) detected.
    \item There need to be a testing mechanism so that bad changes can be
          detected quickly.
  \end{itemize}
}

\frame[containsverbatim]{ \frametitle{Test Cases}
  \begin{itemize}
    \item Code can be $>10$ years old and still very good.
      \begin{itemize}
        \item Cannot rewrite old code every year (and introduce new errors
              every year).
      \end{itemize}
    \item But need to make sure old code is actually still working, despite
          the many other changes to the framework and other components that it
          interacts with.
    \item A \emph{test case} stores program input and expected output so that
          any change in behavior can be detected.
   \item Test cases can also be used to test portability. Should get the same
         result on different architectures to within roundoff error.
  \end{itemize}
}

\frame[containsverbatim]{ \frametitle{Recovering From Errors}
  \begin{itemize}
    \item Mistakes happen (bugs) and it should be possible to undo bad changes
          to the code.
    \item It is important, therefore, to keep the complete history of all
          changes to the code in order to be able to undo changes when
          necessary.
    \item Need to use source code management tools such as subversion, darcs,
          git, mercurial\ldots This not only keeps track of the changes to
          the code but also who made them.
  \end{itemize}
}

\frame[containsverbatim]{ \frametitle{Working Together}
  \begin{itemize}
    \item A source code management system also defines a \emph{single
          standard version} of the components on which everybody is working.
    \item It would be too confusing to send source code around by email or
          look into other directories.
    \item Source code management systems also allows for temporary branches
          for heavy development when adding new features without disturbing
          people doing production runs.
    \item Source code management systems are indispensable for scientific
          code development.
    \item Tutorials for source code management systems are available online.
  \end{itemize}
}

\frame[containsverbatim]{ \frametitle{Policies}
  \begin{itemize}
    \item Working in a group on a code base requires some policies regarding:
      \begin{itemize}
        \item Coding style (routine names, indentation, commit messages).
        \item Access rights (using, modifying, adding, committing).
        \item Testing standards before committing changes.
        \item Peer review before/after making changes.
      \end{itemize}
    \item It is necessary to know what is acceptable behavior.
  \end{itemize}
}

\frame[containsverbatim]{ \frametitle{Component Life Cycle}
  \begin{itemize}
    \item Idea, experimental implementation.
    \item Prototype, useful for a single paper.
    \item Production code, more features added, most bugs removed, useful for
          a series of papers.
    \item Mature code, very useful, few changes.
    \item Outdated, used mostly for historic investigations but still somewhat
          useful.
  \end{itemize}
}

\frame[containsverbatim]{ \frametitle{Portability}
  \begin{itemize}
    \item Machines become old, outdated and unreliable after a few years, while
          new machines become available.
    \item HPC systems frequently (sometimes once a week!) require maintenance
          or a unavailable for longer periods of time for an upgrade (maybe
          once a year!).
    \item Installed software (compilers) may have bugs that makes a machine
          unusable until fixed.
    \item Therefore, scientific codes need to be portable so that one can then
          quickly use other machines.
  \end{itemize}
}

\frame[containsverbatim]{ \frametitle{New Hardware Architectures}
  \begin{minipage}[t]{4cm}
    \vspace{0pt}
    \includegraphics[width=3.8cm]{cpu_frequency}\\
    \includegraphics[width=3.8cm]{power_consumption}
  \end{minipage}
  \begin{minipage}[t]{7.5cm}
    \vspace{0pt}
    \includegraphics[width=7.3cm]{Projected_Performance_Development}\\
  \end{minipage}
}

\frame[containsverbatim]{ \frametitle{New Hardware Architectures}
  \begin{itemize}
    \item Software stays around much longer than hardware.
      \begin{itemize}
         \item Software: $>15$ years (Cactus).
         \item Hardware: 3 years on average (5 years at most).
      \end{itemize}
    \item Software design must not only be portable but also architecture
          independent.
    \item Software has to be adaptable when architecture changes dramatically.
  \end{itemize}
}

\frame[containsverbatim]{ \frametitle{Multi Core CPUs}
  \begin{itemize}
    \item The clock speed has not increased since 2005 whereas the transistor
          density has continued increasing.
    \item End result: more and more cores on a chip resulting in nodes
          with multiple cores able to access the same shared memory.
    \item Could in principle continue to use pure MPI parallelization.
    \item However, remember memory overhead due to ghost zones and additional
          computational overhead from domain decomposition.
    \item Scaling can suffer at very large core counts.
  \end{itemize}
}

\frame[containsverbatim]{ \frametitle{OpenMP}
  \begin{itemize}
    \item Better scaling may be obtained by using OpenMP (Open
          Multi-Processing) parallelization within a shared memory node and
          MPI parallelization between nodes.
    \item OpenMP is a shared memory parallelization paradigm based on
          lightweight threads and workload distribution between threads.
    \item Parallisation is implemented using compiler directives to tell
          the compiler how to distribute the work in a loop among threads. 
    \item Parallelization can be done incrementally (i.e.\ loop by loop).
    \item OpenMP directives ignored when compiling with a sequential compiler.
    \item Works with F77, F90, C and C++.
  \end{itemize}
}

\frame[containsverbatim]{ \frametitle{OpenMP examples}
C:\\
{\footnotesize
\begin{verbatim}
#pragma omp parallel for private(i,w) shared(N,a,b) reduction(+:sum)
for(i = 0; i < N; i++)
{
  b[i] = 2*a[i]+3;
  w = i*i;
  sum = sum + w*a[i];
}
\end{verbatim}
}

Fortran:\\
{\footnotesize
\begin{verbatim}
!$omp parallel do private(i,w) shared(N,a,b) reduction(+:sum)
do i = 1, N
  b(i) = 2*a(i)+3
  w = (i-1)*(i-1)
  sum = sum + w*a(i)
end do
!$omp end parallel do
\end{verbatim}
}
}
 
\frame[containsverbatim]{ \frametitle{Accelerator Computing}
  \begin{minipage}[t]{6.5cm}
    \vspace{-0.5em}
    \begin{itemize}
      \item Started in 2001 with a Sony/Toshiba/IBM collaboration for the
            \emph{Cell} processor.
      \item It has one ``real'' core (Power Processor Element) and 8 
            ``additional'' cores (Synergistic Processing Element) on a single
            chip that are linked together by a high speed bus
            (Element Interconnect Bus).
      \item It is used in Sony's Playstation, Roadrunner at Los Alamos
            National Laboratory (\#19 on the top 500 in 2012) and other places.
      \item Newest player: Intel Xeon Phi used in TACC's Stampede.
    \end{itemize}
  \end{minipage}
  \begin{minipage}[t]{5cm}
    \vspace{0pt}
    \includegraphics[width=4.8cm]{cell-processor}
  \end{minipage}
}
 
\frame[containsverbatim]{ \frametitle{GPU Computing}
  \begin{minipage}[t]{6.5cm}
    \vspace{0pt}
    \begin{itemize}
      \item Graphics cards with a Graphic Processing Unit (GPU) have a similar
            architecture with many cores.
      \item In a GPU each core has to perform the same operation (Single
            Instruction Multiple Data) i.e. the cores together work like a
            vector unit.
      \item The GPU cannot access the main memory of the CPU.
      \item Three out of the top 10 machines on the current top 500 contains
            a significant portion of GPUs.
    \end{itemize}
  \end{minipage}
  \begin{minipage}[t]{5cm}
    \vspace{0pt}
    \includegraphics[width=4.8cm]{gpu-schematic}
  \end{minipage}
}
 
\frame[containsverbatim]{ \frametitle{GPU Computing Challenges}
  \begin{itemize}
    \item Each core is small and slow. A GPU needs to use many cores to be as fast
          as a regular CPU.
      \begin{itemize}
        \item Slower clock speed.
        \item Less memory per core.
      \end{itemize}
    \item Need to use even more cores to get good performance.
    \item But, if this is possible, then total speed is much larger than that
          of regular CPUs (of same total cost) and uses a lot less power.
    \item Memory management is a lot more complicated.
      \begin{itemize}
        \item Data has to be copied from CPU main memory to GPU memory and
              back (slow).
        \item Algorithms have to be modified to do as many calculations
              as possible on data on GPU before copying back and forth.
      \end{itemize}
  \end{itemize}
}
 
\frame[containsverbatim]{ \frametitle{Framework Architecture Challenges}
  \begin{itemize}
    \item Don't want to redesign framework and rewrite code for new 
          architectures.
    \item The framework should isolate the programmer from architectural
          changes as much as possible.
    \item MPI clusters have been around for about 20 years.
    \item Now something new is coming (has already arrived).
      \begin{itemize}
        \item Multicore CPUs (requires use of OpenMP, pthreads or similar).
        \item Cell and GPU accelerators (requires use of CUDA, OpenCL or OpenACC).
	\item Intel Xeon Phi accelerators currently only works with the Intel
	      compilers.
      \end{itemize}
  \end{itemize}
}
 
\frame[containsverbatim]{ \frametitle{Cactus Approach to Architecture
                                      Independence}
  \begin{itemize}
    \item Separate physics code from computer science code.
    \item Each thorn ``sees'' only the small part of the overall problem that
          is relevant for its function (information hiding).
    \item Ideally each physics thorn would act on a single grid point at
          a time. However, that would have too much overhead.
    \item This externalizes parallelism, load balancing, data distribution and
          data sharing (\emph{largest Cactus success}).
    \item This also significantly complicates programming (\emph{largest Cactus
          problem}).
    \item Multicore and accelerator programming poses a distinct challenge for
          Cactus.
    \item Currently being addressed with macros, templates and automatic code
          generation (LoopControl, CaKernel and Kranc). 
  \end{itemize}
}
 
\frame[containsverbatim]{ \frametitle{Scientific Programming Practices Summary}
  \begin{itemize}
    \item Use a (or several) source code management system for code development.
    \item Keep track of software versions for each simulation (Formaline).
    \item Have test cases for each component to ensure correctness.
    \item Need portability and architecture independence or at least enough
          flexibility to make programs future-proof.
  \end{itemize}
}

\end{document}
