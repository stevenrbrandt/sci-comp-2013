\documentclass[11pt]{article}

\usepackage[utf8]{inputenc}
\usepackage{graphicx}
\usepackage{amssymb}
\usepackage{epstopdf}
\usepackage{array}
\usepackage{color}
\usepackage{psfrag}
\usepackage{url}
\pagestyle{empty}
\textwidth = 6.5 in
\textheight = 9 in
\oddsidemargin = 0.0 in
\evensidemargin = 0.0 in
\topmargin = 0.0 in
\headheight = 0.0 in
\headsep = 0.0 in
\parskip = 0.1in
\parindent = 0.0in

\def\journaldata#1#2#3#4{{\it #1} {\bf #2:} #3 (#4)}
 %
 % Usage is {journal name} {volume} {pages} {year}
 % (volume can also be {volume and issue number})

\def\eprint#1{$\langle$#1\hbox{$\rangle$}}
\def\apj{Astrophys.\ J.}

\newcommand{\code}[1]{\textsf{#1}}
\newcommand{\todo}[1]{{\color{blue}$\blacksquare$~\textsf{[TODO: #1]}}}

\begin{document}

\section*{Annual Report: TG-SEE100004}

\paragraph{Annual Report for Period:}
2012 -- 2013

\paragraph{Title:} Graduate Course on ``Scientific Computing''\

\section{Project Participants}

The following personnel participated in the course in the 2012 -- 2013 timeframe.

\subsection{Teaching Personell}

\paragraph{Frank Löffler} (PI) is IT consultant at the Center for Computation
and Technology. He is expert in simulations in numerical relativity. He
leads the Einstein Toolkit and co-manages the frameworks/Cactus group
at LSU. He coordinated the organization of the course and lectured on
Basic Skills.

\paragraph{Steven Brandt} (co-PI) is faculty at the Center for Computation and
Technology and at the department for Computer Science at Louisiana State
University. He co-manages the frameworks/Cactus group. He helped to coordinated
the organization of the course and lectured on advanced programming tools.

\paragraph{Peter Diener} (co-PI) is faculty at the Center for Computation and
Technology and at the department for physics  at Louisiana State University. He
lectured on simulation and application frameworks.

\paragraph{Werner Benger} (co-PI) is IT consultant at the Center for
Computation and Technology, and is specialist in scientific visualization,
on which is also lectured within the course.

\paragraph{Shantenu Jha} (co-PI) is faculty at Rutgers the State University,
but still has an appointment at LSU. He lectured on his primary field of expertise,
distributed scientific computing.

\section{Activities and Findings}

Including a first general lecture introducing the class and setting the scene
for scientific computing, the course was divided into five different modules
(see figure~\ref{fig:overview}),
covering different topics and each taught by a different instructor. Although the
modules were designed to be relatively independent, they were also
designed to be coherent with each other and carry consistent threads.  Each
module consisted of four to seven lectures depending on the needs of the
topic, with each lecture lasting eighty minutes.  The module instructor was
responsible for the module curricula, course work and a component of the final
exam.

\begin{figure}[t]
    \centerline{\includegraphics[height=5cm]{figs/Overview}}
    \caption{\label{fig:overview}
      The five modules for the graduate level Scientific Computing course. While
      Modules B to E have each been taught as block, Module A was interwoven with
      all other modules.}
\end{figure}

We don't provide details about the curricula in this report. Instead, the
interested reader is refered to~\cite{FL-Allen2011a} and in particular in
connection with XSEDE to~\cite{FL-Loeffler2011tg}.

XSEDE provided an essential part of the computing allocation for our graduate
course on scientific computing. Along with the use of real-world research
applications, the national production resources of XSEDE provided students with
experience using a real-world computational environment. The allocation process
for acquiring resources was straightforward, and XSEDE staff were very
responsive in dealing with accounts. One of the major outcomes of the course is
that students were made aware of the breadth of computational facilities
available to academic researchers in the USA and will hopefully continue to
make good use of these during their careers. Through this course, students were
required to face a number of small issues which complicate computational
research, such as local environments, problems with compilation and deployment
etc., and this course hopefully further provided students with the confidence
to work through these problems and to be aware of the options for assistance
through documentation and help desk facilities.  This course will be offered
again at Louisiana State University in Fall 2013.

\bibliographystyle{abbrv-url}
\bibliography{references}

\end{document}

