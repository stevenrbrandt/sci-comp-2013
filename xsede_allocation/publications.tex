\documentclass[11pt]{article}

\usepackage[utf8]{inputenc}
\usepackage{graphicx}
\usepackage{amssymb}
\usepackage{epstopdf}
\usepackage{array}
\usepackage{color}
\usepackage{psfrag}
\usepackage{url}
\pagestyle{empty}
\textwidth = 6.5 in
\textheight = 9 in
\oddsidemargin = 0.0 in
\evensidemargin = 0.0 in
\topmargin = 0.0 in
\headheight = 0.0 in
\headsep = 0.0 in
\parskip = 0.1in
\parindent = 0.0in

\def\journaldata#1#2#3#4{{\it #1} {\bf #2:} #3 (#4)}
 %
 % Usage is {journal name} {volume} {pages} {year}
 % (volume can also be {volume and issue number})

\def\eprint#1{$\langle$#1\hbox{$\rangle$}}
\def\apj{Astrophys.\ J.}

\newcommand{\code}[1]{\textsf{#1}}
\newcommand{\todo}[1]{{\color{blue}$\blacksquare$~\textsf{[TODO: #1]}}}

\begin{document}

\section*{Annual Report: TG-SEE100004}

\paragraph{Publications for Period:}
2010 -- 2013

\paragraph{Title:} Graduate Course on ``Scientific Computing''\\\


\section{Refereed Conference Proceedings}

\begin{itemize}

\item Gabrielle Allen, Werner Benger, Andrei Hutanu, Shantenu Jha, Frank L{\"o}ffler,
  and Erik Schnetter.
  \emph{A practical and comprehensive graduate course preparing students for
    research involving scientific computing},
  Proceedings of the International Conference on Computational Science,
    ICCS 2011, Jan 2011

\item Frank L\"offler, Gabrielle Allen, Werner Benger, Andrei Hutanu, Shantenu Jha,
  and Erik Schnetter.
  \emph{Using the TeraGrid to Teach Scientific Computing},
  TG '11: Proceedings of the 2011 TeraGrid Conference, New York, NY, USA, 2011. ACM

\end{itemize}

\section{Web/Internet Site}

\begin{itemize}
 \item The course material is publicly available at \url{https://svn.cct.lsu.edu/repos/sci-comp/public/}.
\end{itemize}

\section{Posters}

\begin{itemize}
 \item Frank Löffler presented the experiences teaching the graduate course ``Scientific Computing'' at the Technology Share Fair / TechPawLooza at Louisiana State University in 2011 and 2012.
\end{itemize}

\section{Talks}
\begin{itemize}
 \item Frank Löffler presented the experiences of using TeraGrid within the graduate course ``Scientific Computing'' at the TeraGrid 2011 conference.
\end{itemize}

\end{document}

